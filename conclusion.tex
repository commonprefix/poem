\section{Conclusion}

In this paper, we introduced \poem, a new fork choice rule, in which each block
counts for $\work(B) = -\lg \frac{H(B)}{T}$ instead of the usual $\work(B) = \frac{1}{T}$
(Section~\ref{sec:construction}).
We illustrated experimentally (Section~\ref{sec:results}) that \poem achieves better transaction confirmation latency (28.5\% improvement),
or better transaction throughput (16.3\% improvement), for the same level of adversarial resilience ($\beta = \frac{t}{n}$) and level
of security (security parameter $\kappa$). We formally proved the security of \poem (Corollary~\ref{cor:security}) in the Bitcoin Backbone model
(Section~\ref{sec:analysis}).
In our proof, we introduced the novel \emph{real-valued random oracle} model (Section~\ref{sec:model}), which allowed us
to use tools from the continuous domain such as the exponential distribution. We showed this
model to be closely related to the \emph{discrete random oracle} model (Lemma~\ref{lem:hash-separation}),
and believe this new mathematical tooling may be independently useful for the analysis of other protocols.
We reported on the production-grade deployment of our protocol, which has an already deployed testnet
and has seen wide community adoption (Section~\ref{sec:deployment}).
Our protocol only changes the proof-of-work inequality, and thus is composable with a multitude
of previously proposed improvements for latency and throughput in proof-of-work blockchains. We are hopeful
that our modification will be adopted by existing and future proof-of-work protocols in the community.