\section{Resilience Analysis}

Now that we have proven PoEM is secure, we now turn our attention to the benefits of PoEM as compared to the
fork choice rule of traditional proof-of-work. We posit that the intrinsic work fork choice rule
provides better adversarial resilience than the longest (or heaviest) fork choice rule.

For this analysis, we work in a simplified model in which time is continuous
and the cryptographic machinary of hash functions is abstracted away as a perfect
stochastic processes (cf., the previous section where our model was discrete time
and the imperfect stochastic process was constructed by a hash function modelled as
a random oracle). This treatment was introduced in \emph{Everything is a Race}~\cite{eiar},
where it was used to analyse longest-chain protocols, and
allows us to focus on the essence of the problem.

Consider an adversary controlling a $\beta$ fraction of the mining power, whereas the
honest parties control a $1 - \beta$ fraction. Each honest party maintains a local chain,
which is the chain with the most intrinsic work seen by that party so far.
Each honest party continuously attempts to mine blocks on top of their currently
adopted longest chain. As soon as a block is found, this block is broadcast to the
network. The adversary can delay the delivery of these network messages by a duration
up to $\Delta = 1$, and these delays can be different for different honest receipients.
On the contrary, the adversary functions as one overarching entity.
The adversary can mine on top of any chain she has seen so far, and withhold blocks
which can be revealed at a later time. As soon as an honest party sees any
block, the honest party re-broadcasts that block, and this block becomes known to
all honest parties after a $\Delta$ duration. Hence, adversarial blocks revealed to one
honest party become known to the whole honest population within a delay of $\Delta$.

We model this process as a Poisson process with rate $g \in \mathcal{R}^+$ blocks per
unit time for the honest parties and $g \frac{\beta}{1 - \beta}$ for the adversary. We
assume that the population of honest parties is large, such that no honest party
will have two successful events in an execution.

In Bitcoin,
as proven in the \emph{Everything is a Race} work~\cite[Theorem 5.1]{eiar}, there is
an \emph{optimal} adversarial strategy that every other adversarial strategy can be reduced to.
This optimal strategy is the private mining attacker: The adversary and honest parties
all begin at genesis. The adversary mines a private chain, which she never reveals to the honest
parties. The honest parties mine on their own honest subtree. This attack is victorious if,
at the conclusion of the execution, the adversary was able to produce a private chain longer
than the longest chain of any honest party, namely the height of the honest subtree. Since
any other attack can be reduced to this attack, to show security it suffices that the expected
growth rate of the adversary's private chain is less than the expected growth rate of the
height of the honest subtree. In their work, they calculate the honest subtree growth rate to be
$\frac{g}{1 + g}$. By taking the inequality $g \frac{beta}{1 - \beta} > \frac{g}{1 + g}$,
requiring that the honest subtree growth rate is larger than the adversarial private chain
growth rate, in expectation, they find that the maximum adversarial resilience in Bitcoin is
$\beta < \frac{1}{g + 2}$.

We use the same argument here, adapted to the entropic setting: Again, the honest parties mine
blocks following a Poisson process with a block production rate of $g \in \mathbb{R}^+$, whereas
the adversary mines blocks following a Poisson process with a block production rate of
$g \frac{\beta}{1 - \beta}$. Each newly produced valid block $B$ has work $\work(B)$ which follows
an exponential distribution with rate $\ln2$, and all of these works are mutually independent.

\dznote{TODO: Make work ~ exponential + const. Consider different exponential rates, and calculate
the different Chernoff bounds depending on the exp rate.}

The adversary follows a private
mining strategy and the honest parties mine on their own honest subtree. At the conclusion of the
execution, the adversary is victorious if she holds a chain with more intrinsic work than any honest
party.

Consider a protocol execution $\Epsilon$. Let 

\[
  \chi^\Epsilon(t) = \begin{cases}
    0 & \text{ if no honest party found a block at time } t\\
    1 & \text{ if at least one honest party found a block at time } t
  \end{cases}
\]

The function $\chi(t)$ is a Poisson process with rate $g$, where $\chi(t) = 1$
indicates the Poisson successes. For any $t$ such that $\chi(t) = 1$, let $X(t)$ be a
random variable distributed according to the exponential distribution with rate $\ln2$
such that $\{ X(t) \}_{t \in \chi^{-1}(1)}$ are identically distributed and mutually
independent. These $X(t)$ indicate the work of the block found at time $t$.

First, we observe that the expected rate at which the intrinsic work of the private adversarial
chain grows is $g \frac{\beta}{1 - \beta} \frac{1}{\ln2}$. This is because the expected work of a
block is $\frac{1}{\ln2}$, and the adversary produces blocks at a rate of $g \frac{\beta}{1 - \beta}$.
Consider an execution $\Epsilon$ during a snapshot taken at time $t$. Let
$f^\Epsilon(t) = \max_{P \text{ honest}}{\work(C^P_t)}$ be the maximum intrinsic work of any honestly
adopted chain at time $t$. We wish to calculate the expected growth rate of $f^\Epsilon(t)$, namely
$\alpha = \E[\frac{f^\Epsilon(t)}{t}]$ when $\Epsilon$ is sampled from all executions.

For a given interval $[t - dt, t]$, let $X^{dt}(t) = \max X([t - dt, t])$. Note that, if
$dt$ is small enough (namely, shorter than the shortest distance between two successive successes),
then there will only be one $t^*$ in $[t - dt, t]$ for which $X(t^*) > 0$, and $X^{dt}(t)$
will take that value. The $\max$ function is irrelevant when we take $dt \to 0$.

Given an execution $\Epsilon$, consider the family of functions (parameterized by $dt$):

\[
  f^{\Epsilon,dt}(t) = \begin{cases}
             \max\{f^{\Epsilon,dt}(t - 1) + X^{dt}(t),&\\
    \phantom{\max\{\}}      f^{\Epsilon,dt}(t - dt)\} & \text{, if } \exists t^* \in [t - dt, t]: \chi(t^*) = 1\\
          f^{\Epsilon,dt}(t - dt) & \text{ otherwise }
  \end{cases}\,.
\]

It holds that $f^\Epsilon(t) = \lim_{dt \to 0}{f^{\Epsilon,dt}(t)}$.
Let $f(t) = \E[f^\Epsilon(t)]$ with the execution $\Epsilon$ randomly sampled from all executions.
The rate of growth of $f(t)$ is constant, and the function takes the form $f(t) = \alpha t$.
We can use the above family of functions to calculate $\alpha$.

Notice that $\E[\lim_{dt \to 0} f^{\Epsilon,dt}(t)] = \lim_{dt \to 0} \E[f^{\Epsilon,dt}(t)]$.
Therefore it suffices to calculate $\E[f^{\Epsilon,dt}(t)]$ for a given $dt$.
Let $f^{dt}(t) = \E[f^{\Epsilon,dt}(t)]$. For this function, it holds that:

\[
  f^{dt}(t) = \Pr[\exists t^* \in [t - dt, t]: \chi(t^*) = 1](\Pr[f^{dt}(t - 1) + X^{dt}(t) > f^{dt}(t - dt)](f^{dt}(t - 1) + X^{dt}(t)) + (1 - \Pr[f^{dt}(t - 1) + X^{dt}(t) > f^{dt}(t - dt)])(f^{dt}(t - dt)))\\
            + (1 - \Pr[\exists t^* \in [t - dt, t]: \chi(t^*) = 1])(f^{dt}(t - dt)) \\
            % = (1 - e^{-g dt})()
\]