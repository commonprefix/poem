\section{Ethics Considerations}

In this paper, we did not interact with any live systems (in the form of attacks or data collection).
We designed a new protocol and took measurements sourced from two different deployments: Firstly,
a simulation which we ran in our controlled environment; and secondly, a real-world deployment of
a testnet. The simulation is not used by real users beyond our research group. The real-world deployment
of the testnet \emph{is} used by real users, but is explicitly designed for testing purposes and does
not carry any real value. Additionally, all the measurements we took on the testnet are about aggregate
metrics, and not about any individual user; for example, we measured the latency and throughput of the
system. Therefore, no responsible disclosure or informed consent was required.

Our paper improves on the performance of proof-of-work blockchains. Blockchains have been historically used
for a variety of purposes, some of whose ethics are questionable (e.g. gambling, darknet markets, etc.).
We believe improving the underlying infrastructural technology of blockchains is a step in the right direction,
and we hope that this paper will be a useful contribution to that cause. Proof-of-work itself has been criticized
for its energy consumption. However, there is an ongoing discussion on the trade-offs between energy consumption
and security. We believe our contributions will be useful to the ongoing discussion to fairly determine whether
proof-of-work is a good choice for the future of blockchain technology. Because our protocol does not change the
energy requirements of proof-of-work, but improves on latency and throughput, if proof-of-work is to be used,
we believe our protocol is a step in the right direction.

\section{Open Science}

In an effort to support open science, we uploaded a Creative-Commons--licensed early version of our paper to the
ePrint archive and continue to keep it updated. We also made the source code of our simulation available on
GitHub under a permissive license. Lastly, the production implementation of our testnet (and upcoming mainnet)
is fully open source and available on GitHub. For anonymity purposes, we have included anonymized copies of
the GitHub repositories in the paper. Should the reviewers require the source code to be made available in
the form of an artifact, we will be happy to provide it.
