\section{Analysis}\label{sec:analysis}

To analyze the security of PoEM,
we work in the following variant of the Random Oracle model~\cite{ro}, in which
the random oracle operates as follows internally.

\begin{definition}[Real-Valued Random Oracle]
  The \emph{real-valued random oracle} $H$ can be queried with an input value $x$
  and returns the value $y[{:}{\kappa}]$ as follows.
  When queried with $x$ for the first time,
  it samples a real value $y$ uniformly at random from the continuous interval $(0, 1)$,
  and returns its first $\kappa$ bits $H(x) = y[{:}{\kappa}]$.
  It then remembers the pair $(x, y)$. We denote by $\rH(x)$ this sampled value $y$.
  When queried with $x$ for a subsequent time, it returns the first $\kappa$ bits $H(x) = y[{:}{\kappa}]$
  of the stored $y$.
\end{definition}

We note that this definition is equivalent to the standard Random Oracle model,
as the real number $y$ is unobservable by any Turing Machine, and the only
observable quantity is the $\kappa$-bit approximation $y[{:}{\kappa}]$. Despite this,
in the analysis, we will make use of the real-valued random variable $y = \rH(x)$.

\begin{definition}[Intrinsic Work]
  We define the \emph{intrinsic work} of a real number $A \in [0, 1)$ as
  $\work(A) = \gamma - \lg \frac{A[{:}{\kappa}]}{T} = \gamma - \lg \frac{\frac{\lfloor 2^\kappa A \rfloor}{2^\kappa}}{T} \in [\gamma, +\infty]$,
  where $\gamma \in \mathbb{R}^+$ is the \emph{bias} parameter of the protocol.
\end{definition}

We note that the above definition is a generalization of Definition~\ref{def:quantized-block-work}
for $A \in [0, 1)$.

\begin{definition}[Real Intrinsic Work]
  We define the \emph{real intrinsic work} of a real number $A \in [0, 1)$ as
  $\rwork(A) = \gamma - \lg \frac{A}{T} \in [\gamma, +\infty]$,
  where $\gamma \in \mathbb{R}^+$ is the \emph{bias} parameter of the protocol.
\end{definition}

To simplify the analysis, we will set bias $\gamma = 0$.
We observe that, for a block hash $H(x)$,
the actual protocol uses $\work(H(x))$, which is an approximation
of $\rwork(\rH(x))$. Because $T \in \{ 0, \frac{1}{2^\kappa}, \frac{2}{2^\kappa}, \ldots, \frac{2^\kappa - 1}{2^\kappa}\}$,
we have $\work(H(B)) \leftrightarrow \rwork(\rH(B))$.
We will use the latter value in our analysis.
Looking ahead, our goal will be to demonstrate that the small discrepancy
between these two values is immaterial to the protocol's output. The cornerstone of this
result is stated and proven in the technical \emph{Hash Separation} lemma (Lemma~\ref{lem:hash-separation})
in the appendix. The reason why this real-valued random oracle is useful is that it allows us to
borrow tools from analysis to prove statements about the protocol. In particular, the work of a block
is distributed as $\exp(\frac{1}{\ln 2})$. The sum of the works of multiple blocks in a chain has
a variance that can be bounded using Chernoff bounds that would not be available if we were to use the quantized work.

In a similar vein to the block real intrinsic work, we define the \emph{real} intrinsic work $\rwork(C)$ of a chain $C$, which
is approximated by its intrinsic work $\work(C)$:

\begin{definition}[Real Intrinsic Chain Work]
  We define the \emph{real intrinsic work} of a chain $C$ as
  $\rwork(C) = \sum_{B \in C}{\rwork(\rH(B))}$.
\end{definition}

Completely analogously to the chain addressing notation we defined in Section~\ref{sec:construction},
we define the \emph{real} chain addressing notation $[\alpha] \rlhd C$ as follows.

\noindent
\myparagraph[Real blockchain addressing]
Let $[\alpha] \rlhd C = C[i]$
where
$\rwork(C[{:}{i}]) < \alpha \leq \rwork(C[{:}{i + 1}])$.
If $\rwork(C) < \alpha$, then $[\alpha] \rlhd C = \bot$.
If $\alpha < 0$, then $[\alpha] \rlhd C = C[i]$
where
$\rwork(C[{i}{:}]) < -\alpha \leq \rwork(C[{i + 1}{:}])$.
Let $[{\alpha}{:}{\beta}] \rlhd C = C[{i}{:}{j}]$,
where $i$ is the index of $[\alpha] \rlhd C$
and $j$ is the index of $[\beta] \rlhd C$ in $C$.
Let $[{\alpha}{:}] \rlhd C = C[{i}{:}]$,
and $[{:}\beta] \rlhd C = C[{:}{j}]$,
where $i$ and $j$ are defined with respect to $\alpha$ and $\beta$
as above.

The following three chain virtues will be used as intermediate stepping stones
towards proving the security of the protocol.

\begin{definition}[Entropic Growth]
  The \emph{Entropic Growth} property of
  a \poem execution,
  parametrized by the growth interval $s \in \mathbb{N}$
  and the entropic growth velocity $\tau \in \mathbb{R}^+$,
  states that for
  all honest parties $P$ and all rounds $r_1 + s \leq r_2$,
  the chains $C_1, C_2$ of $P$ at rounds $r_1, r_2$ respectively
  satisfy $\rwork(C_2[{|C_1|}{:}]) \geq s \tau$.
\end{definition}

\begin{definition}[Existential Entropic Quality]
  The \emph{Existential Entropic Quality} property of
  a \poem execution, parametrized by the entropic \emph{quality chunk parameter} $\ell \in \mathbb{N}$,
  % and \emph{quality concentration parameter} $\mu \in \mathbb{R}^+$ (with $\ell \mu \geq 1$)
  states that for
  all honest parties $P$ and all rounds $r$,
  the chain $C$ that $P$ adopts at round $r$
  has the property that
  for every $0 \leq \alpha < \work(C) - \ell$,
  there is at least one honestly generated block in the chain
  $[{\alpha}{:}{\alpha + \ell}] \rlhd C$.
\end{definition}

\begin{definition}[Entropic Common Prefix]
  The \emph{Entropic Common Prefix} property of
  a \poem execution, parametrized by the \emph{common prefix parameter} $k \in \mathbb{N}$
  states that for
  all honest parties $P_1, P_2$
  and all rounds $r_1 \leq r_2$,
  the chains $C_1, C_2$ that $P_1, P_2$ adopt at rounds $r_1, r_2$ respectively
  satisfy $[{:}{-k}] \rlhd C_1 \preccurlyeq C_2$.
\end{definition}

The above three properties are proven in Theorems~\ref{thm:entropic-growth},~\ref{thm:common-prefix}, and~\ref{thm:entoropic-quality}
in the appendix. From these three properties, the safety and liveness of the protocol follow in the next two theorems,
which are proven in the appendix.

\begin{restatable}[\poem is Safe]{theorem}{restateSafety}\label{thm:safety}
  Typical executions of \poem are safe.
\end{restatable}

\begin{restatable}[\poem is Live]{theorem}{restateLiveness}\label{thm:liveness}
  Typical executions of \poem are live with parameter $u = \max(\ceil*{\frac{\ell + 2k}{(1 - \epsilon) f} \ln2}, s)$.
\end{restatable}

The security of the protocol follows from the above two theorems:

\begin{restatable}[\poem is Secure]{corollary}{restateSecurity}\label{cor:security}
  \poem is secure with overwhelming probability.
\end{restatable}