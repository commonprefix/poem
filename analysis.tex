\section{Analysis}

\begin{definition}[Entropic Growth]
  The \emph{Entropic Growth} property of
  a \poem execution,
  parametrized by the growth interval $s \in \mathbb{N}$
  and the growth velocity $\tau \in \mathbb{R}^+$,
  states that for
  all honest parties $P$ and all rounds $r_1 + s \leq r_2$,
  the chains $C_1, C_2$ of $P$ at rounds $r_1, r_2$ respectively
  satisfy $\work(C_2[{|C_1|}{:}]) \geq s \tau \lg T$.
\end{definition}

\begin{definition}[Entropic Quality]
  The \emph{Entropic Quality} property of
  a \poem execution, parametrized by the \emph{quality chunk parameter} $\ell \in \mathbb{N}$
  and \emph{quality concentration parameter} $\mu \in \mathbb{R}^+$
  (with $\ell \mu \geq 1$)
  states that for
  all honest parties $P$ and all rounds $r$,
  the chain $C$ of $P$ at round $r$
  has the property that
  for every $0 \leq \alpha < \work(C) - \ell \lg T$,
  there is at least one honestly generated block in the chain
  $[{\alpha}{:}{\alpha + \ell \lg T}] \lhd C$.
\end{definition}

\begin{definition}[Entropic Common Prefix]
  The \emph{Entropic Common Prefix} property of
  a \poem execution, parametrized by the \emph{common prefix parameter} $k \in \mathbb{N}$
  states that for
  all honest parties $P_1, P_2$
  and all rounds $r_1 \leq r_2$,
  the chains $C_1, C_2$ of $P_1, P_2$ at rounds $r_1, r_2$ respectively
  satisfy $[{:}{-k}] \lhd C_1 \preceq C_2$.
\end{definition}

\begin{conjecture}[Entropic Growth]
  Typical executions of \poem satisfy the Entropic Growth property
  with $s = \lambda$ and $\tau = (1 - \epsilon)f$.
\end{conjecture}

\begin{conjecture}[Entropic Quality]
  Typical executions of \poem satisfy the Entropic Quality property
  with $\ell = 2 \lambda f$ and
  $\mu = 1 - (1 + \frac{\delta}{2})\frac{t}{n - t} - \frac{\epsilon}{1 - \epsilon}$.
\end{conjecture}

\begin{conjecture}[Entropic Common Prefix]
  Typical executions of \poem satisfy the Entropic Common Prefix property
  with $k = 2 \lambda f$.
\end{conjecture}

In the analysis we are going to assume honest majority.

\begin{definition}[Honest Majority Assumption]
  We say that an execution has \emph{honest majority} with \emph{honest advantage parameter}
  $0 < \delta \leq 1$, if the number $t$ of corrupted parties out of
  $n$ parties satisfies $t \leq (1 - \delta) (n - t)$.
\end{definition}
\atnote{Add the balancing equation $3f + 3\epsilon < \delta \leq 1$ separately}


Consider an execution of the \poem protocol.

We define a random variable $A_{r, i, j}$ as follows.
If at round $r$, the $j$-th query of (honest or adversarial) party $P_i$ is a valid block $B$,
then $A_{r, i, j} = \work(H(B))$.
If no valid block is found, $A_{r, i, j} = 0$.
Observe that $A_{r, i, j}$ can be expressed in the form $A_{r, i, j} = C_{r, i, j} W_{r, i, j}$,
with independent boolean random variable $C_{r, i, j} \sim \Bern(T)$ indicating whether the query was successful
and real random variable $W_{r, i, j} \sim \Exp(\ln2)$ measuring the work of the block found.

We define $X_{r} = \max_{j=1}^q \max_{i = 1}^{[n - t]} A_{r, i, j}$.
If at round $r$ at least one honest party finds a valid block ($X_r > 0$),
we say that round $r$ is a \emph{successful round}.
Let $\overline{X}_r = \sum_{i = 1}^{n - t} \sum_{j = 1}^q A_{r,i,j}$ and
\[
  \underbar{X}_r = \begin{cases}
  0 \text{, if there are no $i, j$ with $A_{r,i,j} > 0$; otherwise,}\\
  A_{r,i,j} \text{, where $(i, j)$ are the minimum such $(i, j)$.}
\end{cases}\]
Observe that $\underbar{X}_r \leq X_r \leq \overline{X}_r$
and $\underbar{X}_r \sim \Bern(1 - (1 - T)^{q(n - t)}) \Exp(\ln 2)$.

We define a random variable $Y_r$ as follows.
If at round $r$ exactly one honest party obtains a valid block, then $Y_r = X_r$,
and we call $r$ a \emph{convergence opportunity}. Otherwise, $Y_r = 0$.

We define $Z_{r}$ as the sum of all intrinsic work generated by all adversarial
party queries during round $r$. Hence, $Z_{r} = \sum_{i = n - t + 1}^n \sum_{j = 1}^q A_{r, i, j}$.

Given a set of rounds $S$, we define
$X(S) = \sum_{r \in S} X_r$,
$\overline{X}(S) = \sum_{r \in S} \overline{X}_r$,
$\underbar{X}(S) = \sum_{r \in S} \underbar{X}_r$,
$Y(S) = \sum_{r \in S} Y_r$
and $Z(S) = \sum_{r \in S} Z_r$.
Observe $\underbar{X}(S) \leq X(S) \leq \overline{X}(S)$.

We calculate the following expectations:
\begin{align*}
  \E[W_{r, i, j}] = \frac{1}{\ln2}\,.\\
  \E[A_{r, i, j}] =\\
  \E[A_{r, i, j} | C_{r, i, j} = 0] \Pr[C_{r, i, j} = 0] +\\
  \E[A_{r, i, j} | C_{r, i, j} = 1] \Pr[C_{r, i, j} = 1] =\\
  % \E[W_{r, i, j}] * \Pr[C_{r, i, j} = 1]
  \frac{T}{\ln2}\,.\\
  \E[Z(S)] =
  % \E[\sum_{r \in S} Z_r] = \sum_{r \in S} \E[Z_r] = \sum_{r \in S} \sum_{i = 1}^{t}{ \sum_{j = 1}^{q}{ \E[Z_{r, i, j}] } } = tq|S| \E[A_{r, i, j}] =\\
  \frac{tq|S| T}{\ln2}\,.
\end{align*}

The probability of a convergence opportunity is
\begin{align*}
  (n - t) (1 - (1 - T)^q) (1 - T)^{q(n - t - 1)} \geq \\
  q(n - t) T (1 - T)^{q(n - t) - 1} > \\
  q(n - t) T (1 - (q(n - t) - 1)T) > \\
  q(n - t) T (1 - q(n - t)T)\,.
\end{align*}
The first expression is the probability that exactly one honest party is successful;
the second that exactly one query is successful. The penultimate inequality is by
Bernoulli's inequality.
Observe that $Y_r \sim \Bern((n - t)(1 - T)^{(n - t - 1)q}(1 - (1 - T)^q)) \Exp(\ln 2)$.

Therefore,
\begin{align*}
  \E[Y_r] > \frac{q(n - t)T(1 - q(n - T)T)}{\ln2}\,.\\
  \E[\underbar{X}_r] = \frac{1 - (1 - T)^{(n - t)q}}{\ln2}\,.\\
  \E[\overline{X}_r] = \frac{(n - t)qT}{\ln2}\,.
\end{align*}

\begin{definition}[Causality]
  An execution is \emph{causal} if no block (directly or indirectly) extends
  one which is computed at a later or the same random oracle query.
\end{definition}

\begin{definition}[\poem Typical Execution]
  An execution of \poem is \emph{($\epsilon,\lambda$)-typical} (or just typical),
  for $\epsilon \in (0,1)$ and integer $\lambda > 4$, if for any set $S$ of at
  least $\lambda$ consecutive rounds, the following hold.
  \begin{itemize}
    \item $(1 - \epsilon) \E[\underbar{X}(S)] < X(S) < (1 + \epsilon) \E[\overline{X}(S)]$ and $(1-\epsilon) \E[Y(S)] < Y(S)$.\label{item:typicality-x-y}
    \item $Z(S) < (1 + \epsilon)\E[Z(S)]$.\label{item:typicality-z}
    \item It is causal.\label{item:typicality-causal}
    \item It has hash separation.\dznote{Define hash separation}.\label{item:typicality-hash-separation}
  \end{itemize}
\end{definition}

\begin{theorem}[Typicality]
  An execution of duration $L$ of \poem is $(\epsilon, \lambda)$-typical with
  probability $1 - e^{-\Omega(\epsilon^2\lambda f - \log L)} - e^{-\Omega(\kappa - \log L)}$,
  namely, overwhelming in $\lambda$ and $\kappa$.
\end{theorem}
\begin{proof}
  For each $S$ with $|S| = \lambda$,
  \begin{align*}
    \Pr[X(S) < (1 - \epsilon)\E[\underbar{X}(S)]] &\leq\\
    \Pr[\underbar{X}(S) < (1 - \epsilon)\E[\underbar{X}(S)]] &\leq
    e^{-\Omega(\lambda)} \,.\\
    \Pr[X(S) > (1 + \epsilon)\E[\overline{X}(S)]] &\leq\\
    \Pr[\overline{X}(S) > (1 + \epsilon)\E[\overline{X}(S)]] &\leq
    e^{-\Omega(\lambda)} \,.\\
    \Pr[Y(S) < (1 - \epsilon)\E[Y(S)]] \leq e^{-\Omega(\lambda)} &\,.\\
    \Pr[Z(S) > (1 + \epsilon)\E[Z(S)]] \leq e^{-\Omega(\lambda)} &\,.\\
  \end{align*}
  The $e^{-\Omega(\lambda)}$ bounds are obtained by applying
  Theorem~\ref{thm:bern-exp} to each of the random variables
  $\underbar{X}(S), \overline{X}(S), Y(S)$ and $Z(S)$, each
  of which is the sum of $\Theta(\lambda)$ i.i.d. random variables\
  distributed according to $\Bern(p) \times \Exp(\ln2)$ for
  some respective $p \in (0, 1)$.
  Applying a union bound for all $S$ (of which there are $L - \lambda + 1$),
  we obtain that typicality
  points~\ref{item:typicality-x-y} and~\ref{item:typicality-z}
  hold with probability $1 - e^{-\Omega(\lambda)+\ln L}$.
  If typicality points~\ref{item:typicality-x-y} and~\ref{item:typicality-z}
  hold for all $S$ with $|S| = \lambda$, then they hold for all $S$ with
  $|S| \geq \lambda$.

  \dznote{TODO: Prove hash separation.}

  The probability bound for causality follows from the stochastic nature
  of the Random Oracle and is proven in~\cite{backbone}.
  \Qed
\end{proof}

% \begin{theorem}[Concentration of Gamma]
%   Consider a family $\{ X_i \}_{i \in [n]}$ of i.i.d. random variables $X_i$ distributed as $\Exp(\lambda)$, $\lambda > 0$,
%   and let $X = \sum_{i = 1}^{n}{X_i}$. Then, for any $0 < \epsilon < 1$, it holds that
%   $\Pr[X > (1 + \epsilon) \E[X]] < e^{-n(\epsilon - \ln(1 + \epsilon))}$,
%   which is negligible in $n$.
% \end{theorem}
% \begin{proof}
% %  First, we calculate the expectation:
% %
% %  \[
% %    \E[X] = \E[\sum_{i = 1}^n X_i] = \sum_{i = 1}^n \E[X_i] = n \E[X_i] = \frac{n}{\lambda}
% %  \]
% %
% %  Next, we calculate the moment generating function:
% %
% %  \[
% %    \E[e^{tX}] = \E[e^{t \sum_{i = 1}^n} X_i] = \E[\prod_{i = 1}^n e^{t X_i}] = \prod_{i = 1}^n \E[e^{t X_i}]
% %    = \E[e^{t X_i}]^n = (\frac{\lambda}{\lambda - t})^n
% %  \]
% %
% %  The third equality stems from the mutual independence of the family $\{ X_i \}_{i \in [n]}$,
% %  whereas the last equality stems from the moment generating function of the exponential.
%   $X$ is distributed as $\Gamma(\alpha=n, \beta=\lambda)$. Therefore
%   $\E[X] = \frac{n}{\lambda}$ and the moment generating
%   function is $\E[e^{tX}] = (\frac{\lambda}{\lambda - t})^n = e^{n\ln(\frac{\lambda}{\lambda - t})}$.
%
% % For all $0 < t < \lambda$,
% % \begin{align*}
% %   \Pr[X > \alpha] = \Pr[e^{tX} > e^{t\alpha}] \leq \frac{\E[e^{tX}]}{e^{t\alpha}}\,.
% % \end{align*}
%
% %  The first equality is by the fact that the exponential function is increasing,
% %  whereas the second inequality is Markov's inequality.
%   \begin{align*}
%     \Pr[X > (1 + \epsilon)\E[X]] = \Pr[X > (1 + \epsilon)\frac{n}{\lambda}] \leq \E[e^{tX}] e^{-t(1 + \epsilon)\frac{n}{\lambda}}\\
%     = e^{n\ln(\frac{\lambda}{\lambda - t}) - t(1 + \epsilon)\frac{n}{\lambda}}\,.
%   \end{align*}
%
%   The first inequality is by the generic Chernoff bound.
%   To minimize the exponent over $t$, we differentiate it and equate to $0$:
%
%   \begin{align*}
%     \frac{d}{dt}(\ln(\frac{\lambda}{\lambda - t}) - t(1 + \epsilon)\frac{1}{\lambda}) &= 0\\
%     \frac{d}{dt}{\ln\lambda - \ln(\lambda - t) - t(1 + \epsilon)\frac{1}{\lambda}} &= 0\\
%     \frac{1}{\lambda - t} - (1 + \epsilon)\frac{1}{\lambda} &= 0\\
%     \frac{1}{\lambda - t} &= (1 + \epsilon)\frac{1}{\lambda}\\
%     \lambda - t &= \frac{\lambda}{1 + \epsilon}\\
%     t &= \lambda(1 - \frac{1}{1 + \epsilon}) = \frac{\lambda \epsilon}{1 + \epsilon}\,.
%   \end{align*}
%
%   We substitute $t$ in the exponent:
%
%   \begin{align*}
%     n \ln(\frac{\lambda}{\lambda - t}) - t(1 + \epsilon)\frac{n}{\lambda} = \\
%     n \ln(\frac{\lambda}{\lambda - \frac{\lambda\epsilon}{1 + \epsilon}}) - \frac{\lambda \epsilon}{1 + \epsilon}(1 + \epsilon)\frac{n}{\lambda} = \\
%     n \ln(\frac{1}{1 - \frac{\epsilon}{1 + \epsilon}}) - \epsilon n = \\
%     n \ln(\frac{1}{\frac{1 + \epsilon - \epsilon}{1 + \epsilon}}) - \epsilon n = \\
%     n \ln(1 + \epsilon) - \epsilon n = \\
%     -n (\epsilon - \ln(1 + \epsilon))\\
%   \end{align*}
%
%   The exponent is negative because
%   \begin{align*}
%     \epsilon - \ln(1 + \epsilon) > 0 \Leftrightarrow \\
%     \epsilon > \ln(1 + \epsilon)\\
%     e^{\epsilon} > 1 + \epsilon\,,
%   \end{align*}
%   which holds because $\epsilon > 0$.
% \end{proof}

\begin{theorem}[Concentration of $\Bern \times \Exp$]\label{thm:bern-exp}
  Let $\{ A_i \}_{i \in [n]}$ and $\{ B_i \}_{i \in [n]}$ be two families of i.i.d. random variables,
  all mutually independent,
  with $A_i$ distributed as $\Bern(p)$ and $B_i$ distributed as $\Exp(\lambda)$.
  Let $X_i = A_i B_i$, and $X = \sum_{i = 1}^n X_i$.
  Then for any $0 < \epsilon < 1$, it holds that
  $\Pr[X > (1 + \epsilon) \E[X]] < e^{-\Omega(n)}$ and
  $\Pr[X < (1 - \epsilon) \E[X]] < e^{-\Omega(n)}$,
  which is negligible in $n$.
\end{theorem}
\begin{proof}
  $\E[X_i] = \E[A_i B_i] = \E[A_i] \E[B_i] = \frac{p}{\lambda}$, therefore
  $\E[X] = \frac{np}{\lambda}$. For the moment generating functions we have

  \begin{align*}
    &\E[e^{t X_i}] = \E[e^{t A_i B_i}] =\\
      &\E[e^{t A_i B_i}|A_i = 0] \Pr[A_i = 0]\\
    + &\E[e^{t A_i B_i}|A_i = 1] \Pr[A_i = 1] = \\
    \E[e^{t A_i B_i}|A_i = 0] (1 - p) + &\E[e^{t A_i B_i}|A_i = 1] p = \\
    (1 - p) + p \E[e^{t B_i}] &= (1 - p) + p \frac{\lambda}{\lambda - t}\,.
  \end{align*}

  \begin{align*}
    \E[e^{tX}] = \E[e^{t \sum_{i = 1}^n X_i}] = \E[\prod_{i = 1}^n e^{t X_i}] = \prod_{i = 1}^n \E[e^{t X_i}] = \\
    \E[e^{t A_i B_i}]^n = \left[(1 - p) + p\frac{\lambda}{\lambda - t}\right]^n = e^{n \ln\left[(1 - p) + p\frac{\lambda}{\lambda - t}\right]}\,.
  \end{align*}

  For all $0 < t < \lambda$:

  \begin{align*}
    \Pr[X > (1 + \epsilon)\E[X]] = \Pr[X > (1 + \epsilon)\frac{np}{\lambda}]\\
    \leq \E[e^{tX}] e^{-t(1 + \epsilon)\frac{np}{\lambda}}
    = e^{n \ln\left[(1 - p) + p\frac{\lambda}{\lambda - t}\right] - n t(1 + \epsilon)\frac{p}{\lambda}}\,.
  \end{align*}

  Consider the factor
  $f(t) = \ln\left[(1 - p) + p\frac{\lambda}{\lambda - t}\right] - t(1 + \epsilon)\frac{p}{\lambda}$
  in front of $n$ in the exponent. Taking its derivative with respect to $t$:

  \begin{align*}
    \frac{d}{dt} \ln\left[(1 - p) + p\frac{\lambda}{\lambda - t}\right] - t(1 + \epsilon)\frac{p}{\lambda} = \\
    \frac{1}{(1 - p) + p\frac{\lambda}{\lambda - t}} \frac{d}{dt} \left[(1 - p) + p\frac{\lambda}{\lambda - t}\right] - (1 + \epsilon)\frac{p}{\lambda} = \\
    \frac{p\frac{\lambda}{(\lambda - t)^2}}{(1 - p) + p\frac{\lambda}{\lambda - t}} - (1 + \epsilon)\frac{p}{\lambda}
  \end{align*}

  At $t = 0$ we have $f(0) = 0$ and
  \begin{align*}
    f'(0) = \frac{\frac{p}{\lambda}}{(1 - p) + p} - (1 + \epsilon)\frac{p}{\lambda} =\\
    \frac{p}{\lambda}(1 - 1 - \epsilon) = -\frac{\epsilon p}{\lambda} < 0\,.
  \end{align*}

  Since $f'$ is continuous at $0$ and $f'(0) < 0$, there must exist some $0 < t^* < \lambda$ such that for all
  $0 < t < t^*$ it holds that $f'(t) < 0$. Because $f$ is continuous and differentiable in $[0, t^*]$,
  by the Mean Value Theorem, there must exist some $\xi \in (0, t^*)$ such that
  $f'(\xi) = \frac{f(t^*) - f(0)}{t^* - 0} = \frac{f(t^*)}{t^*}$.
  Since $t^* > 0$ and $f'(\xi) < 0$, therefore $f(t^*) < 0$.
  This $t^*$ makes the factor in front of $n$ in the exponent negative, and therefore
  gives us a bound for which $\Pr[X > (1 + \epsilon)\E[X]] < e^{-\Omega(n)}$.

  For all $t < 0$:

  \begin{align*}
    \Pr[X < (1 - \epsilon)\E[X]] = \Pr[X < (1 - \epsilon)\frac{np}{\lambda}]\\
    % = \Pr[e^ {tX} > e^{t(1 - \epsilon)\frac{np}{\lambda}}]\\
    % \leq \E[e^{tX}] e^{-t(1 - \epsilon)\frac{np}{\lambda}} \\
    \leq \E[e^{tX}] e^{-t(1 - \epsilon)\frac{np}{\lambda}} \\
    = e^{n \ln\left[(1 - p) + p\frac{\lambda}{\lambda - t}\right] - n t(1 - \epsilon)\frac{p}{\lambda}}
  \end{align*}

  Consider the factor
  $f(t) = \ln\left[(1 - p) + p\frac{\lambda}{\lambda - t}\right] - t(1 - \epsilon)\frac{p}{\lambda}$
  in front of $n$ in the exponent. Taking its derivative with respect to $t$:

  \begin{align*}
    \frac{d}{dt} \ln\left[(1 - p) + p\frac{\lambda}{\lambda - t}\right] - t(1 - \epsilon)\frac{p}{\lambda} = \\
    \frac{1}{(1 - p) + p\frac{\lambda}{\lambda - t}} \frac{d}{dt} \left[(1 - p) + p\frac{\lambda}{\lambda - t}\right] - (1 - \epsilon)\frac{p}{\lambda} = \\
    \frac{p\frac{\lambda}{(\lambda - t)^2}}{(1 - p) + p\frac{\lambda}{\lambda - t}} - (1 - \epsilon)\frac{p}{\lambda}
  \end{align*}

  At $t = 0$ we have $f(0) = 0$ and
  \begin{align*}
    f'(0) = \frac{\frac{p}{\lambda}}{(1 - p) + p} - (1 - \epsilon)\frac{p}{\lambda} =\\
    \frac{p}{\lambda}(1 - 1 + \epsilon) = \frac{\epsilon p}{\lambda} > 0\,.
  \end{align*}

  Since $f'$ is continuous at $0$ and $f'(0)>  0$, there must exist some $t^* < 0$ such that for all
  $t^* < t < 0$ it holds that $f'(t) > 0$. Because $f$ is continuous and differentiable in $[t^*, 0]$,
  by the Mean Value Theorem, there must exist some $\xi \in (t^*, 0)$ such that
  $f'(\xi) = \frac{f(0) - f(t^*)}{0 - t^*} = \frac{f(t^*)}{t^*}$.
  Since $t^* < 0$ and $f'(\xi) > 0$, therefore $f(t^*) < 0$.
  This $t^*$ makes the factor in front of $n$ in the exponent negative, and therefore
  gives us a bound for which $\Pr[X < (1 - \epsilon)\E[X]] < e^{-\Omega(n)}$.
  \Qed
\end{proof}

\begin{definition}[Block Work Interval]
  A block $B$ of chain $C$ has \emph{work interval}
  $I(B) = \{\xi \geq 0: [\xi] \lhd C = B\}$.
\end{definition}

%TODO: Define convergence opportunity & successful round

\begin{lemma}[Entropic Pairing Lemma] \label{lem:pairing}
  Suppose a block $B$ of a chain $C$ with work interval $I(B)$
  was computed by an honest party in a convergence opportunity.
  For every $\xi \in I(B)$ and every chain $C'$ of the execution,
  block $B' = [\xi] \lhd C'$ is either $B$ or adversarial,
  as long as $B' \neq \bot$.
  %TODO: Add explanation about random oracle observability
\end{lemma}
\begin{proof}
  Consider an execution as in the statement and suppose, towards a contradiction,
  that block $B'$ is not $B$ and is honestly computed.
  %TODO: Maybe define honestly computed block.
  Since $B$ was computed in a convergence opportunity, $B$ and $B'$
  cannot have been computed in the same round. Let $r$ be the earliest round
  on which $B$ or $B'$ was computed. Since it was computed by
  an honest party, at round $r + 1$, every other honest party receives
  a chain with work greater or equal to $\xi$.
  It follows that every block computed
  after round $r$ will be extending a chain with work more than $\xi$.
  If $B$ is computed after round $r$, it holds that $\xi \not \in I$.
  If $B'$ is computed after round $r$, it holds that $B' \neq [\xi] \lhd C'$.
  Both lead to a contradiction. \Qed
\end{proof}

% The following conjecture is likely true and will allow us to tighten the analysis:
%
% \begin{conjecture}[Entropic Pairing Conjecture]
%   Consider any block $B$ of chain $C$ with work interval $I(B)$
%   computed by an honest party during a round $r$ such that
%   for every $B'$ which was honestly computed during round $r$,
%   it holds that $\work(B) \geq \work(B')$.
%   Then, for every $\xi \in I(B) \setminus I(B')$ and every chain $C'$ of the execution,
%   block $B' = [\xi] \lhd C'$ is either $B$ or adversarial,
%   as long as $B' \neq \bot$.
% \end{conjecture}

\begin{lemma}[Entropic Chain Growth Lemma]
  Suppose that at round $r_1$ an honest party has a chain of work $w$.
  Then, by round $r_2 \geq r_1$, every honest party adopts a chain of work at least
  $w + \sum_{r = r_1}^{r_2 - 1}{X_r}$.
\end{lemma}
\begin{proof}
  By induction on $r_2$. For the inductive base ($r_2 = r_1$), observe that
  if at round $r_1$ an honest party has a chain $C$ of work $w$, then
  that party broadcasted $C$ at the end or round $r_1 - 1$. It follows that
  every honest party receives $C$ at round $r_1$ and adopts a chain with
  greater or equal work.

  For the inductive step, note that by the inductive hypothesis,
  every honest party has received a chain of work at least $w' = w + \sum_{r = r_1}^{r_2 - 2}{X_r}$
  by round $r_2 - 1$. When $X_{r_2 - 1} = 0$ the statement follows directly, so assume
  $X_{r_2 - 1} > 0$. Observe that an honest party successfully queried the random oracle
  with a chain of work at least $w' + X_{r_2 - 1}$ and broadcasted it to the network.
  At round $r_2$, every honest party receives the chain and adopts a chain
  of work at least $w' + X_{r_2 - 1} = w + \sum_{r = r_1}^{r_2 - 1}{X_r}$. \Qed
\end{proof}

\begin{lemma}[Entropic Common Prefix Lemma]
  Suppose at round $r$ of a typical execution an honest party has a chain
  $C_1$, while a chain $C_2$ of work at least $\work(C_1)$ is adopted by an honest party.
  Then, $[{:}{-w}] \lhd C_1 \preccurlyeq C_2$ and $[{:}{-w}] \lhd C_2 \preccurlyeq C_1$
  for $w = 2 \lambda f$.
\end{lemma}
\begin{proof}
    Consider an execution as in the statement and suppose,
    towards a contradiction, that $[{:}{-w}] \lhd C_1 \not \preccurlyeq C_2$
    or $[{:}{-w}] \lhd C_2 \not \preccurlyeq C_1$.
    Consider the last block $B^*$ with index $i^*$ on the common prefix of
    $C_1$ and $C_2$ that was computed by an honest party and let $r^*$
    be the round at which it was computed; if no such block exists let $r^* = 0$.
    Define the set of rounds $S = \{i: r^* < i < r\}$. We claim that
    $Z(S) \geq Y(S)$.

    We show this by pairing all work of blocks computed by honest parties during
    convergence opportunities in $S$ with adversarial work computed during $S$.
    Let $\mathcal{Y}(S)$ be the set of honestly produced blocks in convergence opportunities
    during $S$, and $\Xi = \{\xi \in I(B): B \in \mathcal{Y}(S)\}$.

    Note that $\min{\Xi} > \max{I(B^*)}$ because the chain ending in block $B^*$
    was diffused at round $r^*$, and all honestly produced blocks after round $r^*$
    are extending a chain with greater or equal work.
    Also note that $\work(C_1) \geq \max{\Xi}$ and $\work(C_2) \geq \max{\Xi}$ because
    the honest party that computed the chain with work $\max \Xi$ diffused it and any chain adopted
    by honest parties at any later round should have at least $\max \Xi$ work.
    Hence, for every $\xi \in \Xi$ it holds that
    $[\xi] \lhd C_1 \neq \bot$ and $[\xi] \lhd C_2 \neq \bot$.

    We now argue that for every $\xi \in \Xi$ either block $[\xi] \lhd C_1$
    or block $[\xi] \lhd C_2$ is adversarial. If the block lies on the
    common prefix of $C_1$ and $C_2$ ($[\xi] \lhd C_1 = [\xi] \lhd C_2$),
    then it is adversarial by the definition of $B^*$. Otherwise,
    there is one block in $C_1$ and another one in $C_2$, and by
    Lemma~\ref{lem:pairing}, it holds that $[\xi] \lhd C_1$ and
    $[\xi] \lhd C_2$ cannot both be honest.
    This completes the proof of the claim $Z(S) \geq Y(S)$.

    It holds that all chained work $\work(C_2[{i^*} {:}]) \geq w$
    was produced during $S$.
    Hence, from Lemma~\ref{lem:patience}, $|S| > \lambda$ and
    the properties of a typical execution apply.
    Therefore, by Lemma~\ref{lem:typical-bounds},
    $Z(S) < Y(S)$ which contradicts the previous claim. \Qed
\end{proof}

\atnote{Define Chained Work}


\begin{conjecture}[Entropic Patience] \label{lem:patience}
  In a typical execution, any chained work $w \geq 2 \lambda f$ is computed
  in more than $\lambda = \frac{w}{2 f}$ consecutive rounds.
\end{conjecture}

\begin{conjecture}[\poem is Safe]
  Typical executions of \poem are safe.
\end{conjecture}

\begin{conjecture}[\poem is Live]
  Typical executions of \poem are live.
\end{conjecture}

\begin{corollary}[\poem is Secure]
  Typical executions of \poem are secure.
\end{corollary}
\begin{proof}
  Security follows from safety and liveness.
  \Qed
\end{proof}

\begin{lemma}[Block Work Approximation]\label{lem:block-work-approximation}
  In a PoEM execution, the probability that all blocks $B$
  have $\awork(B) - \work(B) < 2^{-\kappa/2}$
  is overwhelming in $\kappa$.
\end{lemma}
\begin{proof}
  Consider the event $E$ in which for all blocks $B$ it holds that
  $H(B) > \frac{1}{2^{\kappa/2}}$.
  Let us calculate the probability of $\lnot E$. For $\lnot E$ to happen,
  at least one block must have $H(B) \leq \frac{1}{2^{\kappa/2}}$.
  For any block $B$, it holds that $\Pr[H(B) \leq \frac{1}{2^{\kappa/2}}] = \frac{1}{2^{\kappa/2}}$ (from the
  uniform distribution of $H(B)$ in the interval $(0, 1)$ due to it being a real-valued random oracle).
  Since there are at most $nqL$ blocks in the execution, we have
  $\Pr[\lnot E] = \Pr[\exists B: H(B) \leq \frac{1}{2^{\kappa/2}}] \leq \sum_B \Pr[H(B) \leq \frac{1}{2^{\kappa/2}}] \leq \frac{nqL}{2^{\kappa/2}}$,
  which is negligible in $\kappa$,
  so $E$ happens with overwhelming probability.

  Consider a block $B$ of the execution, conditioned on the event $E$.
  Then
  \begin{align*}
        &\awork(B) - \work(B) = -\lg \aH(B) - (-\lg H(B))\\
       <& -\lg(H(B) - \frac{1}{2^\kappa}) - (-\lg H(B))\\
    \leq& -\lg{\frac{1}{2^{\kappa/2}} - \frac{1}{2^\kappa}} - (-\lg{\frac{1}{2^{\kappa/2}}})\\
       =& -\lg\frac{1 - 2^{-\kappa/2}}{2^{\kappa/2}} - \frac{\kappa}{2} \\
       =& -\lg{1 - 2^{-\kappa/2}} \leq -\ln{1 - 2^{-\kappa/2}} \leq 2^{-\kappa/2}\,.
  \end{align*}

  The first inequality stems from the fact that $\aH(B)$ and $H(B)$ must
  differ by less than $\frac{1}{2^\kappa}$. The second inequality stems from
  the fact that the function $-\lg(x - \frac{1}{2^\kappa}) - (-\lg x)$ is
  decreasing for $x > \frac{1}{2^\kappa}$.
  \Qed
\end{proof}

\begin{corollary}[Chain Work Approximation]\label{cor:chain-work-approximation}
  In a PoEM execution, the probability that all chains $C$
  have $\awork(C) - \work(C) < Lqn 2^{-\kappa/2}$
  is overwhelming in $\kappa$.
\end{corollary}
\begin{proof}
  Conditioned on the event of Lemma~\ref{lem:block-work-approximation}, for all
  chains $C$ it holds that
  \begin{align*}
    \awork(C) - \work(C) = \sum_{B \in C}{\awork(B) - \work(B)}
    < \sum_{B \in C}{2^{-\kappa/2}} = |C| 2^{-\kappa/2} \leq Lqn 2^{-\kappa/2}\,.
  \end{align*}
  \Qed
\end{proof}

\begin{lemma}\label{lem:good-ranges}
  Consider a causal PoEM execution $\mathcal{E}$ with $n$ parties, $q$ queries per round per party,
  and total lifetime $L$.
  Consider the $j$-th random oracle query in this execution.
  If the query is successful, let $B$ indicate its produced block, let
  $w = \work(B), \aw = \awork(B)$, and let $C$ be the
  chain it extends, let $w_1 = \work(C), \aw_1 = \awork(B)$, and
  $w_1' = \work(CB), \aw_1' = \awork(CB)$. Consider any other chain $C_i$ that appears in the
  execution, and let $w_2 = \work(C_i), \aw_2 = \awork(C_i)$.
  Let $\BADRANGE_{j,i}$ denote the event that
  $w_1 < w_2$, and, furthermore, either
  $w_2 - \frac{nqL}{2^{\kappa/2}} - \frac{1}{2^{\kappa/2}} \leq w_1 + w < w_2$ or
  $w_2 < w_1 + w \leq w_2 + \frac{nqL}{2^{\kappa/2}} + \frac{1}{2^{\kappa/2}}$.
  Let $\BADRANGE$ denote the event that there exists some random oracle query $j$
  and some chain $C_i$ in the execution such that $\BADRANGE_{j,i}$.
  The probability $\Pr[\BADRANGE]$ is negligible in $\kappa$.
\end{lemma}
\begin{proof}
  Given that a $j$-th query takes place, its $w$ is distributed as $\Exp(\ln2)$,
  so
  \begin{align*}
    \Pr[w_2 - \frac{nqL}{2^{\kappa/2}} - \frac{1}{2^{\kappa/2}} \leq w_1 + w < w_2 | w_1 < w_2] = \\
    \Pr[w_2 - w_1 - \frac{nqL}{2^{\kappa/2}} - \frac{1}{2^{\kappa/2}} \leq w < w_2 - w_1 | w_1 < w_2] = \\
    (1 - 2^{-(w_2 - w_1)}) - (1 - 2^{-(w_2 - w_1 - \frac{nqL}{2^{\kappa/2}} - \frac{1}{2^{\kappa/2}})}) = \\
    2^{-(w_2 - w_1 - \frac{nqL}{2^{\kappa/2}} - \frac{1}{2^{\kappa/2}})} - 2^{-(w_2 - w_1)} = \\
    2^{-(w_2 - w_1)} (2^{\frac{nqL}{2^{\kappa/2}} + \frac{1}{2^{\kappa/2}}} - 1) \leq \\
    2^{\frac{nqL}{2^{\kappa/2}} + \frac{1}{2^{\kappa/2}}} - 1 < \\
    \frac{nqL}{2^{\kappa/2}} + \frac{1}{2^{\kappa/2}}\,.
  \end{align*}

  The second relation is from the cumulative distribution function of the exponential distribution;
  the fifth relation is from the conditioning on $w_1 < w_2$, and the last relation is from
  Lemma~\ref{lem:bernoulli}, noting that $\frac{nqL + 1}{2^{\kappa/2}} < 1$.

  Likewise, for the other direction,
  \begin{align*}
    \Pr[w_2 < w_1 + w \leq w_2 + \frac{nqL}{2^{\kappa/2}} + \frac{1}{2^{\kappa/2}} | w_1 < w_2] = \\
    \Pr[w_2 - w_1 < w \leq w_2 - w_1 + \frac{nqL}{2^{\kappa/2}} + \frac{1}{2^{\kappa/2}} | w_1 < w_2] = \\
    (1 - 2^{-(w_2 - w_1 + \frac{nqL}{2^{\kappa/2}} + \frac{1}{2^{\kappa/2}})}) - (1 - 2^{-(w_2 - w_1)}) = \\
    2^{-(w_2 - w_1)} - 2^{-(w_2 - w_1 + \frac{nqL}{2^{\kappa/2}} + \frac{1}{2^{\kappa/2}})} = \\
    2^{-(w_2 - w_1)} (1 - 2^{-\frac{nqL}{2^{\kappa/2}} - \frac{1}{2^{\kappa/2}}}) \leq \\
    1 - 2^{-\frac{nqL}{2^{\kappa/2}} - \frac{1}{2^{\kappa/2}}} < \\
    \frac{nqL}{2^{\kappa/2}} + \frac{1}{2^{\kappa/2}}\,.
  \end{align*}

  \dznote{Why does the last inequality above hold? Maybe we need to augment Lemma~\ref{lem:bernoulli}
  accordingly.}

  Consequently,
  \begin{align*}
    \Pr[\BADRANGE_{j,i}] =\\
    \Pr[\BADRANGE_{j,i}|w_1 < w_2]\Pr[w_1 < w_2] \leq\\
    \Pr[\BADRANGE_{j,i}|w_1 < w_2] =
    \Pr[w_2 - \frac{nqL}{2^{\kappa/2}} - \frac{1}{2^{\kappa/2}} \leq w_1 + w < w_2 | w_1 < w_2] +
    \Pr[w_2 < w_1 + w \leq w_2 + \frac{nqL}{2^{\kappa/2}} + \frac{1}{2^{\kappa/2}} | w_1 < w_2] =
    2 \frac{nqL + 1}{2^{\kappa/2}}\\
  \end{align*}

  Applying a union bound over all the queries $j$ and chains $i$ of the execution, we obtain
  $\Pr[\BADRANGE] \leq 2 nqL \frac{nqL + 1}{2^{\kappa/2}}$, which is negligible in $\kappa$.
  \Qed
\end{proof}

\begin{lemma}[Hash Separation]
  Consider an execution of PoEM. Let $\HS$ be the event that for all two (adversarial or honest) chains
  $C_1, C_2$ appearing in the execution,
  $\work(C_1) < \work(C_2) \Rightarrow \awork(C_1) < \awork(C_2)$.
  Then, the probability that $\lnot \HS$ is negligible in $\kappa$.
\end{lemma}
\begin{proof}
  There are at most $Lqn$ random oracle queries in the execution.
  Consider the event $\CLOSE$ that the statement of Lemma~\ref{lem:block-work-approximation}
  holds.
  Then the statement of Corollary~\ref{cor:chain-work-approximation} also holds.
  From the lemma we know that $\Pr[\CLOSE]$ is overwhelming.
  The following proof will be conditioned on the event $\CLOSE$.

  Let $\HS_j$ denote the event that $\HS$ holds for all chains appearing
  before, or at, the $j$-th random oracle query, with $j = 0$ indicating
  the beginning of the execution. We know that $\HS_0$ always holds by definition.

  We now calculate $\Pr[\lnot \HS_j | \HS_{j - 1} \land \CLOSE]$.
  Consider the $j$-th random oracle query.
  % \begin{align*}
  %   \Pr[\lnot \HS_j | \HS_{j - 1}] =\\
  %   \Pr[\lnot \HS_j | \HS_{j - 1} \land j \text{ query successful}]\Pr[j \text{ query successful}] + \Pr[\lnot \HS_j | \HS_{j - 1} \land j \text{ query unsuccessful}]\Pr[j \text{ query unsuccessful}] =\\
  %   \Pr[\lnot \HS_j | \HS_{j - 1} \land j \text{ query successful}]\Pr[j \text{ query successful}]\\
  % \end{align*}
  If the query was unsuccessful, then $\HS_j$ holds. Otherwise, let $C_1$ be the chain
  that the $j$-th random oracle query extends, let $B$ be the block mined on it,
  let $C'_1 = C_1 B$, and let $w = \work(B), w_1 = \work(C_1), w'_1 = \work(C'_1)$
  and $\aw, \aw_1, \aw'_1$ be the respective approximate works.
  Consider any other chain $C_2$ with work $w_2 = \work(C_2)$
  and approximate work $\aw_2$
  that has already appeared in the execution,
  and consider the event $\FLIP_{C_1,C_2}$ that
  $w'_1 < w_2 \land \aw'_1 \geq \aw_2$ or $w'_1 > w_2 \land \aw'_1 \leq \aw_2$.
  Consider the case $w_1 \geq w_2$. Then, because $w > 0$, then $w_1 + w > w_2$, therefore $w_1' > w_2$.
  Additionally, by $\HS_{j - 1}$ we have $\aw_1 > \aw_2$, therefore $\aw_1 + \aw > \aw_2$, and
  $\aw_1' > \aw_2$. From this, it follows that $\lnot \FLIP_{C_1,C_2}$ holds.
  Thus, it suffices to only consider the situation where $w_1 < w_2$.

  We distinguish two cases:

  \noindent
  \textbf{Case 1: } $w_1 + w < w_2$.
  If $w_1 + w < w_2 - \frac{nqL}{2^{\kappa/2}} - \frac{1}{2^{\kappa/2}}$,
  then
  \begin{align*}
    \aw_1 - \frac{nqL}{2^{\kappa/2}} + w &< w_2 - \frac{nqL}{2^{\kappa/2}} - \frac{1}{2^{\kappa/2}}\\
    \aw_1 + w &< w_2 - \frac{1}{2^{\kappa/2}}\\
    \aw_1 + \aw - \frac{1}{2^{\kappa/2}} &< w_2 - \frac{1}{2^{\kappa/2}}\\
    \aw_1 + \aw &< w_2 \\
    \aw'_1 &< w_2 \leq \aw_2\,.
  \end{align*}
  It follows that $\lnot \FLIP_{C_1,C_2}$.

  The first inequality is obtained from the conditioning on $\CLOSE$,
  noting that $\aw_1 - \frac{nqL}{2^{\kappa/2}} < w_1$ follows from
  $\aw_1 - w_1 < Lqn 2^{-\kappa/2}$ (Corollary~\ref{cor:chain-work-approximation}).
  The third inequality is also obtained from the conditioning on $\CLOSE$,
  noting that $\aw - \frac{1}{2^{\kappa/2}} < w$ follows from
  $\aw - w < 2^{-\kappa/2}$ (Lemma~\ref{lem:block-work-approximation}).

  \dznote{TODO: Finish this case by showing the negligibility of
  $w_1 + w \geq w_2 - \frac{nqL}{2^{\kappa/2}} - \frac{1}{2^{\kappa/2}}$.}

  \noindent
  \textbf{Case 2: } $w_1 + w > w_2$.

  \dznote{TODO: Finish this proof...}

  Then, $\Pr[\lnot \HS | \CLOSE] = \Pr[\lnot \HS_{Lqn} | \CLOSE]$, which is negligible.
  Therefore, $\Pr[\lnot \HS] = \Pr[\lnot \HS | \CLOSE] \Pr[\CLOSE] + \Pr[\lnot \HS | \lnot \CLOSE] \Pr[\lnot \CLOSE] \leq \Pr[\lnot \HS | \CLOSE] + \Pr[\lnot \CLOSE]$, which is negligible as the sum
  of two negligible quantities.
  \Qed
\end{proof}

\begin{lemma}[Bernoulli's Inequality]\label{lem:bernoulli}
  For all $0 < y < 1$, it holds that $2^y - 1 < y$.
\end{lemma}
\begin{proof}
  It suffices to show that $(y + 1)^{1/y} > 2$,
  as this implies that $2^y < y + 1$ and ultimately $2^y - 1 < y$.
  The inequality $(y + 1)^{1/y} > 2$ holds due to Bernoulli's
  inequality ($(1 + x)^r > 1 + rx$ for all $x > 0$ and $r > 1$),
  setting $x = y$ and $r = 1/y$.
  \Qed
\end{proof}

\begin{lemma}
  $f(\kappa) = 2^{\frac{nqL + 1}{2^{\kappa/2}}} - 1$ is negligible.
\end{lemma}
\begin{proof}
  It suffices to show that $f(\kappa) < \frac{nqL + 1}{2^{\kappa/2}}$
  for sufficiently large $\kappa$, since $\frac{nqL + 1}{2^{\kappa/2}}$
  is negligible.
  This follows by applying Lemma~\ref{lem:bernoulli} for $y = \frac{nqL + 1}{2^{\kappa/2}}$.
  Note that, for large enough $\kappa$, the exponent $\frac{nqL + 1}{2^{\kappa/2}}$
  falls within the range $0 < \frac{nqL + 1}{2^{\kappa/2}} < 1$,
  so the lemma is applicable.
  \Qed
\end{proof}