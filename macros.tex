\newcommand{\pow}{\textsc{PoW}\xspace}
\newcommand{\lwrite}{\textsc{write}\xspace}
\newcommand{\lread}{\textsc{read}\xspace}
\newcommand{\net}{\textsc{net}\xspace}
\newcommand{\broadcast}{\textsc{broadcast}\xspace}
\newcommand{\constructor}{\textsc{constructor}\xspace}
\newcommand{\execute}{\textsc{execute}\xspace}
\newcommand{\receive}{\textsc{re}\-\textsc{ceive}\xspace}
\newcommand{\maxvalid}{\textsc{maxvalid}\xspace}
\newcommand{\validate}{\textsc{validate}\xspace}
\newcommand{\intrinsicDifficulty}{\textsf{intrinsicDifficulty}\xspace}
\newcommand{\temp}{\textsf{temp}\xspace}
\newcommand{\tx}{\textsf{tx}\xspace}
\newcommand{\ctr}{\textsf{ctr}\xspace}
\newcommand{\ssum}{\textsf{sum}\xspace}
\newcommand{\mmax}{\textsf{max}\xspace}
\newcommand{\round}{\textsf{round}\xspace}
\newcommand{\HS}{\textsc{HS}\xspace}
\newcommand{\COLL}{\textsc{COLL}\xspace}
\newcommand{\FLIP}{\textsc{FLIP}\xspace}
\newcommand{\CLOSE}{\textsc{CLOSE}\xspace}
\newcommand{\BADRANGE}{\textsc{BADRANGE}\xspace}

\newcommand{\true}{\textsf{true}}
\newcommand{\false}{\textsf{false}}
\newcommand{\maxwork}{\textsf{maxwork}}
\newcommand{\thiswork}{\textsf{thiswork}}

\newcommand{\Break}{\State\textsf{break}}
\newcommand{\Continue}{\State\textsf{continue}}

\newcommand{\poem}{\text{PoEM}\xspace}

\newcommandx{\Ledger}[3][1=,2=,3=]{\prescript{#1}{}{\pmb{L}}^{#2}_{#3}}
\newcommandx{\Chain}[3][1=,2=,3=]{\prescript{#1}{}{\pmb{C}}^{#2}_{#3}}
\newcommandx{\cdf}[1][1=]{{\text{cdf}}_{#1}}
\newcommandx{\pdf}[1][1=]{{\text{pdf}}_{#1}}
\newcommandx{\PDF}[1][1=]{{\text{PDF}}_{#1}}
\newcommand{\concat}{\,\|\,}
\newcommand{\HX}{\hat X}
\newcommand{\getsrandomly}{\overset{\$}{\gets}}
\newcommand{\Epsilon}{\mathcal{E}}
\newcommand{\Exp}{\textsf{Exp}}
\newcommand{\Bern}{\textsf{Bern}}


\newcommandx{\myparagraph}[1][1=]{%
\ifoakland%
  \textbf{#1.}%
\else%
  \textbf{#1}.%
\fi%
}%

\makeatletter
\let\save@mathaccent\mathaccent
\newcommand*\if@single[3]{%
  \setbox0\hbox{${\mathaccent"0362{#1}}^H$}%
  \setbox2\hbox{${\mathaccent"0362{\kern0pt#1}}^H$}%
  \ifdim\ht0=\ht2 #3\else #2\fi
  }
%The bar will be moved to the right by a half of \macc@kerna, which is computed by amsmath:
\newcommand*\rel@kern[1]{\kern#1\dimexpr\macc@kerna}
%If there's a superscript following the bar, then no negative kern may follow the bar;
%an additional {} makes sure that the superscript is high enough in this case:
\newcommand*\wideaccent[2]{\@ifnextchar^{{\wide@accent{#1}{#2}{0}}}{\wide@accent{#1}{#2}{1}}}
%Use a separate algorithm for single symbols:
\newcommand*\wide@accent[3]{\if@single{#2}{\wide@accent@{#1}{#2}{#3}{1}}{\wide@accent@{#1}{#2}{#3}{2}}}
\newcommand*\wide@accent@[4]{%
  \begingroup
  \def\mathaccent##1##2{%
%Enable nesting of accents:
    \let\mathaccent\save@mathaccent
%If there's more than a single symbol, use the first character instead (see below):
    \if#42 \let\macc@nucleus\first@char \fi
%Determine the italic correction:
    \setbox\z@\hbox{$\macc@style{\macc@nucleus}_{}$}%
    \setbox\tw@\hbox{$\macc@style{\macc@nucleus}{}_{}$}%
    \dimen@\wd\tw@
    \advance\dimen@-\wd\z@
%Now \dimen@ is the italic correction of the symbol.
    \divide\dimen@ 3
    \@tempdima\wd\tw@
    \advance\@tempdima-\scriptspace
%Now \@tempdima is the width of the symbol.
    \divide\@tempdima 10
    \advance\dimen@-\@tempdima
%Now \dimen@ = (italic correction / 3) - (Breite / 10)
    \ifdim\dimen@>\z@ \dimen@0pt\fi
%The bar will be shortened in the case \dimen@<0 !
    \rel@kern{0.6}\kern-\dimen@
    \if#41
      #1{\rel@kern{-0.6}\kern\dimen@\macc@nucleus\rel@kern{0.4}\kern\dimen@}%
      \advance\dimen@0.4\dimexpr\macc@kerna
%Place the combined final kern (-\dimen@) if it is >0 or if a superscript follows:
      \let\final@kern#3%
      \ifdim\dimen@<\z@ \let\final@kern1\fi
      \if\final@kern1 \kern-\dimen@\fi
    \else
      #1{\rel@kern{-0.6}\kern\dimen@#2}%
    \fi
  }%
  \macc@depth\@ne
  \let\math@bgroup\@empty \let\math@egroup\macc@set@skewchar
  \mathsurround\z@ \frozen@everymath{\mathgroup\macc@group\relax}%
  \macc@set@skewchar\relax
  \let\mathaccentV\macc@nested@a
%The following initialises \macc@kerna and calls \mathaccent:
  \if#41
    \macc@nested@a\relax111{#2}%
  \else
%If the argument consists of more than one symbol, and if the first token is
%a letter, use that letter for the computations:
    \def\gobble@till@marker##1\endmarker{}%
    \futurelet\first@char\gobble@till@marker#2\endmarker
    \ifcat\noexpand\first@char A\else
      \def\first@char{}%
    \fi
    \macc@nested@a\relax111{\first@char}%
  \fi
  \endgroup
}
\makeatother
%%%%%%%%%%%%%%%%%%%%%%%%%%%%%%%%%%%%%%%%%%%%%%%%%%%%%%%%%%%%%%%%%%%

\newcommand\doubleoverline[1]{\overline{\overline{#1}}}

\newcommand\widebar{\wideaccent\overline}
\newcommand\widebarbar{\wideaccent\doubleoverline}

\newcommand{\work}{\textsc{work}\xspace}
\newcommand{\rwork}{\widebarbar{\textsc{work}}\xspace}
\newcommand{\rH}{\widebarbar{H}}
\newcommand{\rw}{\widebarbar{w}}
\newcommand{\rlhd}{\widebarbar{\lhd}}

\newcommand{\borderline}{\textsc{Borderline}}