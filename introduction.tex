\section{Introduction}

In Bitcoin~\cite{bitcoin}, a population of \emph{miners}
attempt to find \emph{blocks} in the form $B = h \concat x \concat ctr$,
where $h$ is a pointer to the previous block, $x$ contains a sequence of
\emph{transactions}, and $ctr$ is a \emph{nonce}. The nonce is brute forced
by the miner to satisfy the \emph{proof-of-work} inequality $H(B) \leq T$, where
$H$ is a hash function and $T$ is a small \emph{target}. Any $B$ that satisfies this
inequality is considered a valid block, whereas candidates that do not satisfy
the inequality are invalid. Simply put, valid blocks must have
hashes that begin with a desired number of $0$s.
These blocks form linked lists known as \emph{chains}.
Among such chains, the \emph{longest}\footnote{The
longest chain is chosen in the \emph{static population setting}, which is the
setting in which we work in this paper. In the real deployment of Bitcoin, the
difficulty is dynamically adjusted, and the \emph{heaviest chain} is chosen (c.f.,
~\cite{varbackbone}).} chain is chosen as the canonical one.

Some blocks $B$ satisfy the proof-of-work inequality better than others.
Namely, they satisfy not only $H(B) \leq T$, but also $H(B) \leq \frac{T}{2^\mu}$
for some $\mu \in \mathbb{R}^+$, $\mu > 1$. Nevertheless, this information is ignored when choosing
which chain to pick. We posit that this \emph{information}
produced during the mining process corresponds to \emph{removing entropy} from the
state space of the system (i.e., during mining, the set of future possible states
of the execution as a whole shrinks).
By modifying the fork choice rule of Bitcoin to take into account this information,
we build a protocol which retains the provable security of Bitcoin, while achieving
efficiency advantages over it: (a) it allows reducing the $k$ confirmation parameter,
improving latency, and (b) in a multi-chain setting, it allows increasing the throughput
of the system. In this paper, we prove that the modified protocol is secure in the
single-chain setting.

\noindent
\textbf{Related work.}
Bitcoin was first proven secure in the static population setting~\cite{backbone},
and later also studied under in variable population setting~\cite{varbackbone}.
The idea of using a more nuanced proof-of-work inequality in which some blocks
are considered heavier than others was first put forth by Andrew Miller~\cite{highway},
with the first complete protocol to utilize it being
Proofs of Proof-of-Work~\cite{popow}. These were later refined multiple times
to account for non-interactivity~\cite{nipopows}, backwards compatibility~\cite{velvet-nipopows},
onlineness~\cite{logspace}, on-chain data efficiency~\cite{compact-superblocks},
on-chain consumption~\cite{gasefficient-nipopows},
bribing resilience~\cite{soft-power},
and variable populations~\cite{dionyziz}.
We are the first to modify the fork choice rule to take these refinements into
account. Alternative mechanisms towards improving the latency and throughput
of proof-of-work blockchains at the consensus
layer include parallel chains~\cite{parallel-chains},
separation of transaction/consensus blocks~\cite{prism},
hybrid approaches between proof-of-work and proof-of-stake~\cite{byzcoin},
and the use of microblocks~\cite{bitcoin-ng}. Several of these mechanisms
are orthogonal and can be combined with our approach.
