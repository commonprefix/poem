\section{Introduction}

In Bitcoin~\cite{bitcoin}, a population of \emph{miners}
attempt to find \emph{blocks} in the form $B = h \concat x \concat ctr$,
where $h$ is a pointer to the previous block, $x$ contains a sequence of
\emph{transactions}, and $ctr$ is a \emph{nonce}. The nonce is brute forced
by the miner to satisfy the \emph{proof-of-work} inequality $H(B) \leq T$, where
$H$ is a hash function (with $\kappa$ bits of output)
and $T$ is a small \emph{target}. Any $B$ that satisfies this
inequality is considered a valid block, whereas candidates that do not satisfy
the inequality are invalid. Simply put, valid blocks must have
hashes that begin with a desired number of $0$s.
These blocks form linked lists known as \emph{chains}.
Among such chains, the \emph{longest}\footnote{The
longest chain is chosen in the \emph{static population setting}, which is the
setting in which we work in this paper. In the real deployment of Bitcoin, the
difficulty is dynamically adjusted, and the \emph{heaviest chain} is chosen (c.f.,
~\cite{varbackbone}).} chain is chosen as the canonical one.

Some blocks $B$ satisfy the proof-of-work inequality better than others.
Namely, they satisfy not only $H(B) \leq T$, but also $H(B) \leq \frac{T}{2^\mu}$
for some $\mu \in \mathbb{R}^+$. Nevertheless, these heavier blocks are ignored when choosing
which chain to pick. We posit that this \emph{information}
produced during the mining process corresponds to \emph{removing entropy} from the
state space of the system (i.e., during mining, the set of future possible states
of the execution as a whole shrinks).
By modifying the fork choice rule of Bitcoin to take this into account,
we build a protocol which retains the same level of security as Bitcoin, while achieving
better confirmation latency, because the block production rate can be safely increased.
We report on our production implementation of a real-world system
in which we employ this new rule, and show that it achieves a $2\times$ \atnote{fix this number}
improvement in confirmation latency by measuring the confirmation delay and security
of an experimental deployment of our system.
Additionally, we theoretically prove our new protocol is secure and calculate the safe
bounds within which block production can be increased.

\noindent
\textbf{Construction overview.}
The classical Bitcoin protocol uses the longest chain fork choice. Among two different
chains, the one with the most blocks is chosen as the canonical one. We modify this
rule as follows: Each block counts for its intrinsic work $-\frac{\lg H(B)}{T}$,
where $H(B)$ denotes the hash of the block, and $T$ denotes the fixed\footnote{Our
analysis is in the \emph{static population} model in which the difficulty
and target remain static. In the static model, Bitcoin uses the \emph{longest chain rule}~\cite{backbone}.
On the contrary, in the real deployment of Bitcoin, the difficulty is dynamically adjusted (the
\emph{variable population} model~\cite{varbackbone}), and the \emph{heaviest chain} is chosen.
The scoring in the variable difficulty model makes each block count for $\frac{1}{T}$, where
$T$ is the nominal target of the block. In our protocol, we count the \emph{intrinsic work}
of each block, which is different from the nominal target $T$ and depends on the value $H(B) \leq T$.
We perform our analysis in the static population model, and leave the analysis in the variable
population model for future work.} target. Intuitively, this corresponds to counting the
``number of zeroes'' at the front of the hash of each block. These zeroes are already guaranteed
to be at least $-\lg{T}{2^\kappa}$, but can be more. The score of each block is \emph{how many extra zeroes}
it has. The chain with the most total intrinsic work is chosen as the canonical chain.

To see the benefits of this rule, first observe that it provides a natural tie-breaking rule:
If two honest parties observe the same block tree, they will agree on
the canonical chain, regardless of the order of network message arrival. But the protocol
really shines when the block production rate is increased to its limits. It is generally desirable
to increase the block production rate, because it decreases the confirmation latency of transactions.
However, when the expected block interarrival time approaches the network delay, multiple honest blocks
can be produced in short succession without allowing for honest nodes to synchronize their views.
As a result, multiple honest nodes may produce blocks at the same height, and only one of these will
eventually make it to the canonical chain, while the others will be discarded. These discarded blocks
do not contribute to the growth of the length of the honest chain. Our protocol is similar to Bitcoin in that,
whenever multiple honest blocks are found simultaneously, only one of them makes it to the canonical chain,
and the rest are discarded. However, the surviving block is the heaviest block among them, and the discarded
work is less, leading to a better honest chain intrinsic work growth rate. This gives an advantage to the honest
parties when they are racing against a private mining attacker, since they can produce chains that grow
at a faster rate. This intuition generalizes to arbitrary adversaries.
The small change we introduce in the fork choice rule allows us to safely increase the block production
rate, as compared to Bitcoin, without sacrificing adversarial resilience. The resulting protocol
allows for faster transaction confirmation.

\noindent
\textbf{Related work.}
Bitcoin was first proven secure in the static population setting~\cite{backbone},
and later also studied under in the variable population setting~\cite{varbackbone}.
The idea of using a more nuanced proof-of-work inequality in which some blocks
are considered heavier than others was first put forth by Andrew Miller~\cite{highway},
with the first complete protocol to utilize it being
Proofs of Proof-of-Work~\cite{popow}. These were later refined multiple times
to account for non-interactivity~\cite{nipopows}, backwards compatibility~\cite{velvet-nipopows},
onlineness~\cite{logspace}, on-chain data efficiency~\cite{compact-superblocks},
gas consumption~\cite{gasefficient-nipopows},
bribing resilience~\cite{soft-power},
and variable populations~\cite{dionyziz}.
We are the first to modify the fork choice rule to take these refinements into
account. Previous modifications to the fork choice rule include the GHOST
protocol~\cite{ghost}.
Alternative mechanisms towards improving the latency and throughput
of proof-of-work blockchains at the consensus
layer include parallel chains~\cite{parallel-chains},
separation of transaction/consensus blocks~\cite{prism},
hybrid approaches between proof-of-work and proof-of-stake~\cite{byzcoin},
and the use of microblocks~\cite{bitcoin-ng}. These mechanisms
are orthogonal and can be combined with our approach, yielding even
further performance gains.
