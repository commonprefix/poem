\section{Security Analysis Proofs}\label{app:proofs}

\restateExpectationBounds*
\begin{proof}
  Observe that $A_{r, i, j}$ can be expressed in the form $A_{r, i, j} = C_{r, i, j} W_{r, i, j}$,
  with independent boolean random variable $C_{r, i, j} \sim \Bern(T)$ indicating whether the query was successful
  and real random variable $W_{r, i, j} \sim \Exp(\ln2)$ measuring the real work of the block found.
  Concerning expectations, $\E[W_{r, i, j}] = \frac{1}{\ln2}$, and, furthermore,
  $\E[A_{r, i, j}] = \E[A_{r, i, j} | C_{r, i, j} = 0] \Pr[C_{r, i, j} = 0] +
    \E[A_{r, i, j} | C_{r, i, j} = 1] \Pr[C_{r, i, j} = 1] = \E[A_{r, i, j} | C_{r, i, j} = 1] \Pr[C_{r, i, j} = 1] = \frac{T}{\ln2}$.
  The following bounds are similar to~\cite{backbone}.

  \noindent
  \textbf{Bounds for $f$.}
  We note that
  $\frac{f}{1 - f} = \frac{1 - (1 - T)^{q(n - t)}}{(1 - T)^{q(n - t)}} = (1 - T)^{-q(n - t)} - 1 > (1 + T)^{q(n - t)} - 1 > Tq(n - t)$.
  Here, the penultimate inequality stems from $(1 - T)^{-q(n - t)} > (1 + T)^{q(n - t)} \Leftarrow (1 - T)^{-1} > 1 + T \Leftarrow 1 - T^2 < 1 \Leftarrow
  T > 0$. The last inequality stems from Bernoulli's inequality,
  namely $(1 + x)^r \geq 1 + rx$ for integer $r \geq 1$ and real $x \geq -1$.

  \noindent
  \textbf{Bounds for $\E[X]$.}
  The expectation
  $\E[\underbar{X}_r] = \frac{1 - (1 - T)^{q(n - t)}}{\ln2}$
  follows from fact that
  $\underbar{X}_r \sim \Bern(1 - (1 - T)^{q(n - t)}) \Exp(\ln 2)$.
  The bound on $\E[\underbar{X}_r]$ follows from the previous
  bound on $f$.
  The expectation $\E[\overline{X}_r] = \frac{Tq(n - t)}{\ln2} < \frac{f}{1 - f}\frac{1}{\ln2}$
  follows from the fact that $\overline{X}_r$ is the sum of $q(n - t)$ independent
  random variables distributed as $\Bern(T)\Exp(\ln2)$ and the above bounds on $f$.

  \noindent
  \textbf{Bounds for $\E[Y]$.}
  The probability of a convergence opportunity is
  $(n - t) (1 - (1 - T)^q) (1 - T)^{q(n - t - 1)} \geq T q(n - t) (1 - T)^{q(n - t) - 1} >
  T q(n - t) (1 - (q(n - t) - 1)T) > T q(n - t) (1 - T q(n - t))$.
  The first expression is the binomial probability that exactly one, among $n - t$,
  honest party is successful;
  the second is the binomial probability that exactly one, among $q(n - t)$, honest query is successful,
  which implies that exactly one honest party was successful.
  The penultimate inequality is by
  Bernoulli's inequality, namely $(1 + x)^r \geq 1 + rx$ for integer $r \geq 1$ and real $x \geq -1$.

  We have $Y_r \sim \Bern((n - t)(1 - (1 - T)^q)(1 - T)^{q(n - t - 1)}) \Exp(\ln 2)$,
  therefore, $\E[Y_r] > \frac{T q(n - t)(1 - T q(n - t))}{\ln2} \geq \frac{f(1 - f)}{\ln2} > \frac{(1 - \frac{\delta}{3})f}{\ln2}$.
  For the inequality concerning $\E[Y_r]$, the derivation is analogous to~\cite{backbone}.

  \noindent
  \textbf{Bounds for $\E[Z]$.}
  The expectation $\E[Z_r] = \frac{tqT}{\ln2}$ follows from the fact that $Z_r$
  is distributed as a sum of $tq$ independent samples distributed as $\Bern(T) \Exp(\ln2)$.
  For the bound, we have
  $\E[Z_r] < \frac{t}{n - t}\frac{f}{1 - f}\frac{1}{\ln2} < \left(1 + \frac{\delta}{2}\right)\frac{t}{n - t} \cdot \frac{f}{\ln2}$ % \stepcounter{equation}\tag{\theequation}\label{eq.ex-z-bound}$
    % \E[Z(S)] =
    % \E[\sum_{r \in S} Z_r] = \sum_{r \in S} \E[Z_r] = \sum_{r \in S} \sum_{i = 1}^{t}{ \sum_{j = 1}^{q}{ \E[Z_{r, i, j}] } } = tq|S| \E[A_{r, i, j}] =\\
    % \frac{tq|S| T}{\ln2}\,.
  % \end{align*}
  using an analysis completely analogous to the one in~\cite{backbone}.
  \dznote{Write this derivation explicitly}

  For $\E[Z_r] < \E[X_r]$, we have
  $\E[Z_r] < \E[X_r]
  \Leftarrow \E[Z_r] < \E[\underbar{X}_r]
  \Leftarrow \frac{tqT}{\ln2} < \frac{(1 - f)Tq(n - t)}{\ln2}
  \Leftarrow \frac{t}{n - t} < 1 - f
  \Leftarrow 1 - \delta < 1 - f
  \Leftarrow f < \delta$, which follows from
  the secure configuration.
\end{proof}

\restateBlockWorkApproximation*
\begin{proof}
  Consider the event $E$ in which for all blocks $B$ it holds that
  $\rH(B) > 2^{-(\kappa/2 - 2)}$.
  Let us calculate the probability of $\lnot E$. For $\lnot E$ to happen,
  at least one block must have $\rH(B) \leq 2^{-(\kappa/2 - 2)}$.
  For any block $B$, it holds that $\Pr[\rH(B) \leq 2^{-(\kappa/2 - 2)}] = 2^{-(\kappa/2 - 2)}$ (from the
  uniform distribution of $\rH(B)$ in the interval $(0, 1)$ due to it being a real-valued random oracle).
  Since there are at most $nqL$ blocks in the execution, by applying a union bound, we have
  $\Pr[\lnot E] = \Pr[\exists B: \rH(B) \leq 2^{-(\kappa/2 - 2)}] \leq \sum_B \Pr[\rH(B) \leq 2^{-(\kappa/2 - 2)}] \leq nqL 2^{-(\kappa/2 - 2)}$,
  which is negligible in $\kappa$,
  so $E$ happens with overwhelming probability.

  Consider a block $B$ of the execution, conditioned on the event $E$.
  Then
  \begin{align*}
        &\work(B) - \rwork(B) = -\lg H(B) - (-\lg \rH(B))\\
        \numrel{<}{lem:block-work-approximation-floor}& \lg \rH(B) - \lg\left(\rH(B) - 2^{-\kappa}\right)
        \numrel{<}{lem:block-work-approximation-log-dec} \lg{2^{-(\kappa/2 - 2)}} - \lg\left(2^{-(\kappa/2 - 2)} - 2^{-\kappa}\right)\\
       =& -\left(\frac{\kappa}{2} - 2\right) - \lg(2^{-(\kappa/2 - 2)}(1 - 2^{-(\kappa/2 + 2)})) \\
       =& -\left(\frac{\kappa}{2} - 2\right) + \left(\frac{\kappa}{2} - 2\right) - \lg\left(1 - 2^{-(\kappa/2 + 2)}\right)
       = -\lg\left(1 - 2^{-(\kappa/2 + 2)}\right)\\
       =& -\ln\left(1 - 2^{-(\kappa/2 + 2)}\right) \cdot \frac{1}{\ln2}
    \numrel{\leq}{lem:block-work-approximation-log-ineq} -\frac{-2^{-(\kappa/2 + 2)}}{1 - 2^{-(\kappa/2 + 2)}} \cdot \frac{1}{\ln2} \\
    \leq& \frac{2^{-(\kappa/2 + 2)}}{1 - \frac{1}{2}} \cdot \frac{1}{\ln2}
    = 2^{-(\kappa/2 + 1)} \cdot \frac{1}{\ln2}
    < 2^{-\kappa/2}\,.
  \end{align*}

  Inequality~(\ref{lem:block-work-approximation-floor}) stems from the fact that
  $\rH(B) - 2^{-\kappa} < H(B) \leq \rH(B)$ and the monotonicity of $\lg$.
  Inequality~(\ref{lem:block-work-approximation-log-dec}) stems from
  the fact that the function $\lg x - \lg\left(x - 2^{-\kappa}\right) = \lg \frac{x}{x - 2^{-\kappa}}$ is
  decreasing for $x > 2^{-\kappa}$, remembering our conditioning on
  $\rH(B) > 2^{-(\kappa/2 - 2)} > 2^{-\kappa}$.
  Inequality~(\ref{lem:block-work-approximation-log-ineq}) stems from the standard logarithm
  inequality $\ln(1 + x) \geq \frac{x}{1 + x}$ for all $x > -1$.
  \Qed
\end{proof}

\restateGoodRanges*
\begin{proof}
  If the $j$-th query does take place, its $\rw$ is distributed as $\Exp(\ln2)$,
  so for every other chain $C_i$ in the execution for which $\rw_1 < \rw_2$ we have
  \begin{align*}
    \Pr[\rw_2 - \frac{nqL}{2^{\kappa/2}} - \frac{1}{2^{\kappa/2}} \leq \rw_1 + \rw < \rw_2 | \rw_1 < \rw_2] = \\
    \Pr[\rw_2 - \rw_1 - \frac{nqL}{2^{\kappa/2}} - \frac{1}{2^{\kappa/2}} \leq \rw < \rw_2 - \rw_1 | \rw_1 < \rw_2] = \\
    (1 - 2^{-(\rw_2 - \rw_1)}) - (1 - 2^{-\left(\rw_2 - \rw_1 - \frac{nqL}{2^{\kappa/2}} - \frac{1}{2^{\kappa/2}}\right)}) = \\
    2^{-\left(\rw_2 - \rw_1 - \frac{nqL}{2^{\kappa/2}} - \frac{1}{2^{\kappa/2}}\right)} - 2^{-(\rw_2 - \rw_1)} = \\
    2^{-(\rw_2 - \rw_1)} (2^{\frac{nqL}{2^{\kappa/2}} + \frac{1}{2^{\kappa/2}}} - 1) \leq
    2^{\frac{nqL}{2^{\kappa/2}} + \frac{1}{2^{\kappa/2}}} - 1 <
    \frac{nqL}{2^{\kappa/2}} + \frac{1}{2^{\kappa/2}}\,.
  \end{align*}

  The second relation is from the cumulative distribution function of the exponential distribution;
  the fifth relation is from the conditioning on $\rw_1 < \rw_2$, and the last relation is from
  Lemma~\ref{lem:bernoulli}, noting that $0 < \frac{nqL + 1}{2^{\kappa/2}} < 1$.

  Similarly, for the other direction,
  \begin{align*}
    \Pr[\rw_2 < \rw_1 + \rw \leq \rw_2 + \frac{nqL}{2^{\kappa/2}} + \frac{1}{2^{\kappa/2}} | \rw_1 < \rw_2] = \\
    \Pr[\rw_2 - \rw_1 < \rw \leq \rw_2 - \rw_1 + \frac{nqL}{2^{\kappa/2}} + \frac{1}{2^{\kappa/2}} | \rw_1 < \rw_2] = \\
    (1 - 2^{-\left(\rw_2 - \rw_1 + \frac{nqL}{2^{\kappa/2}} + \frac{1}{2^{\kappa/2}}\right)}) - (1 - 2^{-(\rw_2 - \rw_1)}) = \\
    2^{-(\rw_2 - \rw_1)} - 2^{-\left(\rw_2 - \rw_1 + \frac{nqL}{2^{\kappa/2}} + \frac{1}{2^{\kappa/2}}\right)} = \\
    2^{-(\rw_2 - \rw_1)} (1 - 2^{-\frac{nqL}{2^{\kappa/2}} - \frac{1}{2^{\kappa/2}}}) \leq
    1 - 2^{-\frac{nqL}{2^{\kappa/2}} - \frac{1}{2^{\kappa/2}}} <
    \frac{nqL}{2^{\kappa/2}} + \frac{1}{2^{\kappa/2}}\,.
  \end{align*}

  Consequently,
  \begin{align*}
      &\Pr[\BADRANGE_{j,i}] =\\
      &\Pr[\BADRANGE_{j,i}|\rw_1 < \rw_2]\Pr[\rw_1 < \rw_2]\\
    \leq & \Pr[\BADRANGE_{j,i}|\rw_1 < \rw_2]\\
    = &\Pr[\rw_2 - \frac{nqL}{2^{\kappa/2}} - \frac{1}{2^{\kappa/2}} \leq \rw_1 + \rw < \rw_2 | \rw_1 < \rw_2] +\\
      &\Pr[\rw_2 < \rw_1 + \rw \leq \rw_2 + \frac{nqL}{2^{\kappa/2}} + \frac{1}{2^{\kappa/2}} | \rw_1 < \rw_2]\\
    = &2 \frac{nqL + 1}{2^{\kappa/2}}\\
  \end{align*}

  Applying a union bound over all the queries $j$ and chains $i$ of the execution, we obtain
  $\Pr[\BADRANGE] \leq 2 (nqL)^2 \cdot \frac{nqL + 1}{2^{\kappa/2}}$, which is negligible in $\kappa$.
  \Qed
\end{proof}

\restateHashSeparation*
\begin{proof}
  % There are at most $Lqn$ random oracle queries in the execution.
  Consider a causal execution of PoEM for which
  the event $\CLOSE$ of Lemma~\ref{lem:block-work-approximation}
  and the event $\lnot \BADRANGE$ of Lemma~\ref{lem:good-ranges}
  both hold.
  Observe that the statement of Corollary~\ref{cor:chain-work-approximation}
  holds in this conditioning.
  From the two lemmas we know $\Pr[\CLOSE]$ and
  $\Pr[\lnot \BADRANGE]$ are both overwhelming, therefore
  $\Pr[\CLOSE \land \lnot \BADRANGE]$ is overwhelming.
  Conditioned on this event, we will
  show that the desired statement holds with probability $1$.

  Let $\HS$ be the event that Hash Separation holds.
  Let $\HS_j$ denote the predicate that $\HS$ holds for all chains appearing
  before, or at, the $j$-th random oracle query, with $j = 0$ indicating
  the beginning of the execution.
  We will use induction on $j$ to show that for all $0 \leq j \leq Lnq$,
  $\HS_j$ holds. We know that $\HS_0$ always holds by definition.

  Now, consider the $j$-th random oracle query and suppose $\HS_{j - 1}$ holds.
  If the query was unsuccessful, then $\HS_j$ holds, and we are done.
  Otherwise, let $C_1$ be the chain
  that the $j$-th random oracle query extends, let $B$ be the block mined on it,
  let $C'_1 = C_1 B$, and let $\rw = \rwork(B), \rw_1 = \rwork(C_1), \rw'_1 = \rwork(C'_1)$
  and $w, w_1, w'_1$ be the respective discrete works.
  Consider any other chain $C_2$ with work $\rw_2 = \rwork(C_2)$
  and discrete work $w_2$
  that has already appeared in the execution,
  and consider the undesirable event $\FLIP_{C_1,C_2}$ that
  $\rw'_1 < \rw_2 \land w'_1 \geq w_2$ or $\rw'_1 > \rw_2 \land w'_1 \leq w_2$.
  If $\rw_1 \geq \rw_2$, then, because $\rw > 0$, therefore $\rw_1 + \rw > \rw_2$ and hence $\rw_1' > \rw_2$.
  Additionally, by $\HS_{j - 1}$ we have $w_1 > w_2$, therefore $w_1 + w > w_2$, and
  $w_1' > w_2$. From this, it follows that $\rw_1 \geq \rw_2$ yields $\lnot \FLIP_{C_1,C_2}$.
  Thus, it suffices to only consider the situation where $\rw_1 < \rw_2$.

  \textbf{Case 1: } $\rw_1 + \rw < \rw_2$.
  From the conditioning on $\lnot \BADRANGE$, we have
  $\rw_1 + \rw < \rw_2 - \frac{nqL}{2^{\kappa/2}} - \frac{1}{2^{\kappa/2}}$,
  therefore
  $
    w_1 - \frac{nqL}{2^{\kappa/2}} + \rw < \rw_2 - \frac{nqL}{2^{\kappa/2}} - \frac{1}{2^{\kappa/2}}\Rightarrow
    w_1 + \rw < \rw_2 - \frac{1}{2^{\kappa/2}}\Rightarrow
    w_1 + w - \frac{1}{2^{\kappa/2}} < \rw_2 - \frac{1}{2^{\kappa/2}}\Rightarrow
    w_1 + w < \rw_2 \Rightarrow
    w'_1 < \rw_2 \leq w_2
  $.
  The first inequality is obtained from the conditioning on $\CLOSE$,
  noting that $w_1 - \frac{nqL}{2^{\kappa/2}} < \rw_1$ follows from
  $w_1 - \rw_1 < Lqn 2^{-\kappa/2}$ (Corollary~\ref{cor:chain-work-approximation}).
  The third inequality is also obtained from the conditioning on $\CLOSE$,
  noting that $w - \frac{1}{2^{\kappa/2}} < \rw$ follows from
  $w - \rw < 2^{-\kappa/2}$ (Lemma~\ref{lem:block-work-approximation}).
  It follows that $\lnot \FLIP_{C_1,C_2}$.

  \textbf{Case 2: } $\rw_1 + \rw > \rw_2$.
  From the conditioning on $\lnot \BADRANGE$, we have
  $\rw_1 + \rw > \rw_2 + \frac{nqL}{2^{\kappa/2}} + \frac{1}{2^{\kappa/2}} > \rw_2 + \frac{nqL}{2^{\kappa/2}}$,
  therefore
  $\rw_1 + \rw > w_2 - \frac{nqL}{2^{\kappa/2}} + \frac{nqL}{2^{\kappa/2}} \Rightarrow \rw_1 + \rw > w_2 \Rightarrow
    w_1 + w > w_2 \Rightarrow w'_1 > w_2$.
  Again, it follows that $\lnot \FLIP_{C_1,C_2}$.

  From this and $\HS_{j-1}$ it follows that $\HS_j$ holds.
  Therefore, by induction, $\HS_{Lnq}$ holds, and hence $\HS$ holds.
  Since our conditioning was on an overwhelming event, the lemma follows.
  \Qed
\end{proof}

\restateTypicality*
\begin{proof}
  For each $S$ with $|S| = \lambda$,
  \begin{align*}
    \Pr[X(S) < (1 - \epsilon)\E[\underbar{X}(S)]] &\leq\\
    \Pr[\underbar{X}(S) < (1 - \epsilon)\E[\underbar{X}(S)]] &\leq
    e^{-\Omega(\lambda)} \,.\\
    \Pr[X(S) > (1 + \epsilon)\E[\overline{X}(S)]] &\leq\\
    \Pr[\overline{X}(S) > (1 + \epsilon)\E[\overline{X}(S)]] &\leq
    e^{-\Omega(\lambda)} \,.\\
    \Pr[Y(S) < (1 - \epsilon)\E[Y(S)]] \leq e^{-\Omega(\lambda)} &\,.\\
    \Pr[Z(S) > (1 + \epsilon)\E[Z(S)]] \leq e^{-\Omega(\lambda)} &\,.\\
  \end{align*}
  The $e^{-\Omega(\lambda)}$ bounds are obtained by applying
  Lemma~\ref{lem:bern-exp} to each of the random variables
  $\underbar{X}(S), \overline{X}(S), Y(S)$ and $Z(S)$, each
  of which is the sum of $\Theta(\lambda)$ i.i.d. random variables\
  distributed according to $\Bern(p) \times \Exp(\ln2)$ for
  some respective $p \in (0, 1)$.
  Applying a union bound for all $S$ (of which there are $L - \lambda + 1$),
  we obtain that typicality Eq.~\ref{eq.typical-x}, Eq.~\ref{eq.typical-y}
  and Eq.~\ref{eq.typical-z}
  hold with probability $1 - e^{-\Omega(\lambda)+\ln L}$.
  If typicality bounds
  hold for all $S$ with $|S| = \lambda$, then they hold for all $S$ with
  $|S| \geq \lambda$.

  The probability bound for causality follows from the stochastic nature
  of the Random Oracle and is proven in~\cite{backbone}.

  Lastly, Hash Separation follows from Lemma~\ref{lem:hash-separation}.
  \Qed
\end{proof}

\restatePairing*
\begin{proof}
  Consider an execution as in the statement and suppose, towards a contradiction,
  that block $B'$ is not $B$ and is honestly computed.
  Since $B$ was computed in a convergence opportunity, $B$ and $B'$
  cannot have been computed in the same round. Let $r$ be the earliest round
  on which $B$ or $B'$ was computed, and $C$ be the chain whose tip this block is.
  Since it was computed by
  an honest party, at round $r + 1$, every other honest party receives
  a chain with real work greater or equal to $\xi$.

  \textbf{Claim: } Every block computed after round $r$ will be extending a
  chain with real work at least $\xi$. To see this, consider a chain $C^*$ that an honest
  party is extending after $r$. Since the party has adopted $C^*$, by the heaviest
  chain rule, $\work(C^*) \geq \work(C)$. By Hash Separation, $\rwork(C^*) \geq \rwork(C) \geq \xi$.

  If $B$ is computed after round $r$, it holds that $\xi \not \in I$ (noting that $\rwork(B) > 0$).
  If $B'$ is computed after round $r$, it holds that $B' \neq [\xi] \lhd C'$.
  Both lead to a contradiction. \Qed
\end{proof}

\restateChainGrowth*
\begin{proof}
  By induction on $r_2$. For the inductive base ($r_2 = r_1$), observe that
  if at round $r_1$ party $P_1$ has a chain $C$ of work $\rw$, then
  $P_1$ broadcasted $C$ at the end or round $r_1 - 1$.
  Party $P_2$ receives $C$ at round $r_1$.
  Consider the chain $C_2$ that $P_2$ adopts at $r_1$.
  Due to the heaviest chain rule, $\work(C_2) \geq \work(C)$,
  therefore, by Hash Separation, $\rwork(C_2) \geq \rw$,
  and the statement follows.

  For the inductive step, note that by the inductive hypothesis,
  every honest party has adopted a chain of real work at least $\rw' = \rw + \sum_{r = r_1}^{r_2 - 2}{X_r}$
  at round $r_2 - 1$. When $X_{r_2 - 1} = 0$ the statement follows directly, so assume
  $X_{r_2 - 1} > 0$. Observe that an honest party $P_3$ successfully queried the random oracle
  at round $r_2 - 1$
  and obtained a chain $C_3$ of real work at least $\rw' + X_{r_2 - 1}$ and broadcasted it to the network.
  At round $r_2$, party $P_2$ receives $C_3$ and
  adopts a chain $C_2$, with $\work(C_2) \geq \work(C_3)$ due to the heaviest chain rule.
  By Hash Separation, $\rwork(C_2) \geq \rwork(C_3) \geq \rw' + X_{r_2 - 1} = \rw + \sum_{r = r_1}^{r_2 - 1}{X_r}$.
  \Qed
\end{proof}

\restateTypicalBounds*
\begin{proof}
  \textbf{Proposition \ref{eq.typ-bound-z}.}
  \begin{align*}
    Z(S) &< (1 + \epsilon) \E[Z(S)] = (1 + \epsilon) \E[Z_r] |S| \\
         &= \E[Z_r] |S| + \epsilon \E[Z_r] |S|\\
         &< \frac{t}{n - t} \cdot \frac{f}{1 - f} \cdot \frac{|S|}{\ln2} + \epsilon \frac{t}{n - t} \cdot \frac{f}{1 - f} \cdot \frac{|S|}{\ln2} \\
         &< \frac{t}{n - t} \cdot \frac{f}{1 - f} \cdot \frac{|S|}{\ln2} + \epsilon f \frac{|S|}{\ln2} \\
         &= \left(\frac{t}{n - t} \cdot \frac{1}{1 - f} + \epsilon \right) f \frac{|S|}{\ln2}
         \leq \left(1 - \frac{2 \delta}{3}\right) f \frac{|S|}{\ln2}\,.
  \end{align*}

  The first relation follows from Definition~\ref{def:typicality} Eq.~\ref{eq.typical-z},
  the second from the independence of $Z_r$, the fourth from
  the bound in Lemma~\ref{lem:expectation-bounds} Eq.~\ref{eq.ex-z-bound},
  the fifth and the last from the bounds in~\cite[Section 13.2.2]{blockchain-foundations}.

  \textbf{Proposition \ref{eq.typ-bound-z-x}.}
  $
    Z(S) < \frac{t}{n - t} \cdot \frac{f}{1 - f} \cdot \frac{|S|}{\ln2} + \epsilon f \frac{|S|}{\ln2}
        %  &< \left(1 + \frac{\delta}{2}\right) \cdot \frac{t}{n - t} \cdot f \frac{|S|}{\ln2} + \epsilon f \frac{|S|}{\ln2}\\
        %  &= \left(1 + \frac{\delta}{2}\right) \cdot \frac{t}{n - t} \E[\underbar{X}(S)] + \epsilon f \frac{|S|}{\ln2}\\
         < \left(1 + \frac{\delta}{2}\right) \cdot \frac{t}{n - t} X(S) + \frac{\epsilon f |S|}{\ln2}
  $.
  \atnote{Do the math instead of citing backbone.}
  The first relation follows from part~(\ref{eq.typ-bound-z}) of this proof,
  and the second from the bound in \cite[Lemma 11(c)]{backbone}.

  \textbf{Proposition \ref{eq.typ-bound-y-z}.}
  $Y(S) > (1 - \epsilon)\E[Y(S)]
         > \left(1 - \frac{\delta}{3}\right) f \frac{|S|}{\ln2}
         > \left(1 - \frac{2\delta}{3}\right) f \frac{|S|}{\ln2}
         > Z(S)$.
  The first inequality follows from Definition~\ref{def:typicality} Eq.~\ref{eq.typical-y},
  the second from the bound in Lemma~\ref{lem:expectation-bounds} Eq.~\ref{eq.ex-y-bound},
  and the last one from part~(\ref{eq.typ-bound-z}) of this proof.
  \Qed
\end{proof}

\restateEntropicGrowth*
\begin{proof}
  Consider a typical \poem execution in which an honest party has a chain $C_1$
  at round $r_1$ and adopts a chain $C_2$ at round $r_2 \geq r_1 + s$.
  Let $S = \{r_1, \ldots, r_2 - 1\}$. Then $|S| \geq s = \lambda$ and,
  applying Definition~\ref{def:typicality} we obtain $X(S) > (1 - \epsilon)\E[\underline{X}(S)]$.
  By
  Lemma~\ref{lem:expectation-bounds}, $\E[\underline{X}(S)] \geq \frac{f}{\ln2}|S|$.
  Hence, $X(S) > (1 - \epsilon)\frac{f}{\ln2}|S|$.
  By Lemma~\ref{lem:chain-growth}, $\rwork(C_2) \geq \rwork(C_1) + X(S)$, as desired.
  \Qed
\end{proof}


\restatePatience*
\begin{proof}
  Assume, towards a contradiction, there is a set of consecutive rounds $S'$ in which
  the chained real work $k$ was computed and $|S'| \leq \frac{k - 8}{2 \frac{f}{1 - f} \frac{1}{\ln2}}$.
  It holds that $X(S') + Z(S') \geq k$.
  Then, there is a set $S \supseteq S'$ of consecutive rounds with
  $|S| = \ceil*{\frac{k - 8}{2 \frac{f}{1 - f} \frac{1}{\ln2}}} + 1 < \frac{k - 8}{2 \frac{f}{1 - f} \frac{1}{\ln2}} + 2$
  such that $X(S) + Z(S) \geq X(S') + Z(S') \geq k$. However,
  because $|S| > \lambda$, typicality applies and from Lemma~\ref{lem:typical-bounds} we obtain
  $X(S) < (1 + \epsilon) \E[\overline{X}(S)] \leq (1 + \epsilon) \E[\overline{X}_r] |S| <
  (1 + \epsilon) \frac{f}{1 - f} \frac{|S|}{\ln2}$
  and
  $Z(S) < (1 + \epsilon) \E[Z(S)] \leq (1 + \epsilon) \E[Z_r] |S| <
  (1 + \epsilon) \frac{t}{n - t} \frac{f}{1 - f}\frac{|S|}{\ln2} <
  (1 + \epsilon) (1 - \delta) \frac{f}{1 - f}\frac{|S|}{\ln2}$.
  Hence,
  \begin{align*}
    X(S) + Z(S) < (1 + \epsilon) \frac{f}{1 - f} \frac{|S|}{\ln2} (1 + 1 - \delta) <
    2 \frac{f}{1 - f} \frac{|S|}{\ln2} <\\
    2\frac{k - 8}{2 \frac{f}{1 - f} \frac{1}{\ln2}} \frac{f}{1 - f} \frac{1}{\ln2} + 4 \frac{f}{1 - f} \frac{1}{\ln2} =\\
    k - 8 + 4 \frac{f}{1 - f} \frac{1}{\ln2} = k - 4 \left(2 - \frac{f}{1 - f} \frac{1}{\ln2}\right) < k\,.
  \end{align*}
  The second inequality follows from the fact that $\epsilon =\frac{\delta}{6} \Rightarrow (1 + \epsilon)(2 - \delta) < 2$.
  The last inequality follows from $f < \frac{1}{2} \Rightarrow 2 - \frac{f}{1 - f} \frac{1}{\ln2} < 0$.
  \Qed
\end{proof}