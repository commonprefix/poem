\section{Definitions \& Model}

\noindent
\textbf{Notation.}
Given a sequence $Y$, we address it using $Y[i]$ to mean the $i^\text{th}$ element (starting from $0$).
We use $|Y|$ to denote the length of $Y$.
Negative indices address elements from the end, so $Y[-i]$ is the $i^\text{th}$ element from
the end, and $Y[-1]$ in particular is the last. We use $Y[i{:}j]$ to denote the subarray of $Y$
consisting of the elements indexed from $i$ (inclusive) to $j$ (exclusive). The notation $Y[i{:}]$ means the
subarray of $Y$ from $i$ onwards, while $Y[{:}j]$ means the subsequence of $Y$ up to (but not including) $j$.
The notation $\concat$ denotes the concatenation of two strings.
Given a sequence of strings $(Y_i)_{i \in [n]}$ we denote by $\big\lVert_{i \in [n]} Y_i$ the concatenation
of all the strings in the sequence, in order. We denote by $\Bern(p)$ the Bernoulli distribution with parameter $p$,
and $\Exp(\lambda)$ the exponential distribution with mean $\frac{1}{\lambda}$.

\begin{definition}[Distributed Ledger Protocol]
  A \emph{distributed ledger protocol} is an Interactive Turing Machine (ITM)
  which exposes the following methods:

  \begin{itemize}
    \item $\lwrite(\tx)$:
      Takes user input by accepting some transaction \tx.
    \item $\lread()$:
      Produces user output in the form of a \emph{ledger} (a sequence of transactions)
  \end{itemize}
\end{definition}

The distributed ledger protocol is executed by a set of $n$ parties.
In a distributed ledger protocol execution, the notation
$\Ledger[P][][r]$ denotes the output of $\lread()$
invoked on party $P$ at the end of round $r$.
We denote that ledger
$\Ledger[P_1][][r_1]$ is a prefix of ledger $\Ledger[P_2][][r_2]$,
using the notation
$\Ledger[P_1][][r_1] \preccurlyeq \Ledger[P_2][][r_2]$. When
$(\Ledger[P_1][][r_1] \preccurlyeq \Ledger[P_2][][r_2]) \lor (\Ledger[P_2][][r_2] \preccurlyeq \Ledger[P_1][][r_1])$ holds,
we use the notation $\Ledger[P_1][][r_1] \sim \Ledger[P_2][][r_2]$.

\begin{definition}[Safety]
  A distributed ledger protocol is \emph{safe} if
  for any honest parties $P_1, P_2$ and any rounds $r_1, r_2$, it holds that
  $\Ledger[P_1][][r_1]~\sim~\Ledger[P_2][][r_2]$.
\end{definition}

\begin{definition}[Liveness]
  A distributed ledger protocol is \emph{live}$(u)$ if
  for any honest party that attempts to inject a transaction $\tx$
  at round $r$, it holds that $\tx \in \Ledger[P][][r+u]$
  for all honest parties $P$.
\end{definition}

\begin{definition}[Secure]
  A distributed ledger protocol is \emph{secure} if it is
  both safe and live$(u)$.
\end{definition}

We analyze the protocol using the model introduced in the Bitcoin Backbone~\cite{backbone} paper.
The protocol execution commences in discrete rounds $1, 2, \ldots$, and has a total duration of
$L$, polynomial in the security parameter $\kappa \in \mathbb{N}$.
We assume a synchronous communication network, where all honestly produced
messages are delivered with a delay of one round (messages sent in a round
arrive at the beginning of the next round).
We also assume a static setting, where the protocol is executed by
a fixed total number of $n \in \mathbb{N}$ parties, unknown to the honest parties.
In the execution, the adversary controls $t < n$ of the parties,
and each of the $n - t$ other parties are honest and execute the prescribed Distributed Ledger
Protocol. We let the first $1, 2, \ldots, n - t$ parties be the honest parties
and the last $n - t + 1, \ldots, n$ parties be the corrupted parties, which may behave arbitrarily.
This choice is without loss of generality~\cite[Proposition 18]{backbone}.
Parties communicate through an unauthenticated network,
meaning that the adversary can ``spoof''~\cite{douceur2002sybil}
the source address of any message that is delivered.

We work in the following variant of the Random Oracle model~\cite{ro}, in which
the random oracle returns a \emph{real number} instead of $\kappa$ bits of output.
One technicality with this model is that the real-number cannot directly be returned
to the machine invoking the oracle. Instead, we allow the querying machine to choose which
bit of the number to obtain.

\begin{definition}[Real-Valued Random Oracle]
  The \emph{real-valued random oracle} $H$ can be queried with a value $x$ and a bit index $j$.
  When queried with $x$ for the first time,
  it samples a real value $y$ uniformly at random from the continuous interval $(0, 1)$,
  and returns its $j$-th bit from the binary representation of the real number $y$.
  It then remembers the pair $(x, y)$.
  When queried with $x$ for a subsequent time, it returns the $j$-th bit
  of the stored $y$.
\end{definition}

We denote by $H(x)[{i}{:}{j}]$ the query that obtains the slice of $H(x)$ from bit index $i$ to $j$.
We liberally use the rest of the previously introduced slicing notation with the hash output,
implying that the random oracle is queried with the desired number of bits.
In particular, we will write $\aH(x)$ to denote $H(x)[{:}{\kappa}]$.

\noindent
\textbf{The $q$-bounded model.}
Following the tradition of the Bitcoin Backbone~\cite{backbone} paper,
during each round, each honest party is allowed to query the random oracle with $q$
different $x$ values. Similarly, the adversary is allowed to query the random oracle
with $t q$ different $x$ values.