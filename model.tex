\section{Model}

We analyze the protocol using the model introduced in the Bitcoin Backbone~\cite{backbone} paper.
The protocol execution commences in discrete rounds $1, 2, \ldots$, and has a total duration of
$L$, polynomial in the security parameter $\kappa \in \mathbb{N}$.
We assume a synchronous communication network, where all honestly produced
messages are delivered with a delay of one round (messages sent in a round
arrive at the beginning of the next round).
We also assume a static setting, where the protocol is executed by
a fixed number of $n$ parties, unknown to the honest parties.
The adversary controls $t$ of the parties.
In a single round, all parties are allowed the same number $q$ of queries to
a random oracle~\cite{ro} $H$. Hence, the adversary can perform $t q$ queries
per round.
Parties communicate through an unauthenticated network,
meaning that the adversary can ``spoof''~\cite{douceur2002sybil}
the source address of any message that is delivered.
