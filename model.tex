\section{Model}

We analyze the protocol using the model introduced in the Bitcoin Backbone~\cite{backbone} paper.
The protocol execution commences in discrete rounds $1, 2, \ldots$, and has a total duration of
$L$, polynomial in the security parameter $\kappa \in \mathbb{N}$.
We assume a synchronous communication network, where all honestly produced
messages are delivered with a delay of one round (messages sent in a round
arrive at the beginning of the next round).
We also assume a static setting, where the protocol is executed by
a fixed number of $n \in \mathbb{N}$ parties, unknown to the honest parties.
The adversary controls $t < n$ of the parties.
In a single round, all parties are allowed the same number $q$ of queries to
a random oracle~\cite{ro} $H$. Hence, the adversary can perform $t q$ queries
per round.
Parties communicate through an unauthenticated network,
meaning that the adversary can ``spoof''~\cite{douceur2002sybil}
the source address of any message that is delivered.

We work in the following variant of the Random Oracle model~\cite{ro}, in which
the random oracle returns a \emph{real number} instead of $\kappa$ bits of output.
One technicality with this model is that the real-number cannot directly be returned
to the machine invoking the oracle. Instead, we allow the querying machine to choose which
bit of the number to obtain.

\begin{definition}[Real-Valued Random Oracle]
  The \emph{real-valued random oracle} $H$ can be queried with a value $x$ and a bit index $j$.
  When queried with $x$ for the first time,
  it samples a value $y$ uniformly at random from the continuous interval $(0, 1)$,
  and returns its $j$-th bit from the binary representation of the real number $y$.
  It then remembers the pair $(x, y)$.
  When queried with $x$ for a subsequent time, it returns the $j$-th bit
  of the stored $y$.
\end{definition}

We denote by $H(x)[{i}{:}{j}]$ the query that obtains the slice of $H(x)$ from bit index $i$ to $j$.
We liberally use the rest of the previously introduced slicing notation with the hash output,
implying that the random oracle is queried with the desired number of bits.