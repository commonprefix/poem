\section{Construction}\label{sec:construction}
In the \poem construction, only the fork choice rule
of the original Bitcoin protocol is modified.
Honest parties, instead of adopting the longest chain, at the beginning
of each round, now adopt the chain with the most intrinsic work.

\import{./}{algorithms/algorithm-honest.tex}

A block is any triplet of the
form $B = (h, x, \ctr)$, where $h \in \{0,1\}^\kappa$, $x \in \{0, 1\}^*$, and $\ctr \in \N$.
The hash of block $B$ is denoted as $H(B) = H(h \concat x \concat \ctr)$.
The $h$ is a reference to the previous block hash $H(B')$.

A \emph{chain} $C$ is a sequence of blocks. The chains considered by honest
parties begin with a designated
block $G$, the \emph{genesis block}. The genesis block is a constant block
known to all parties at the beginning of the execution,
which we consider honest by definition.

When a chain $C$ appears in the execution,
we say that block $C[j]$ \emph{extends} block $C[i]$ if $i < j$.

\begin{definition}[Block Intrinsic Work]\label{def:quantized-block-work}
  The \emph{intrinsic work} of a block hash
  $A \in \{0, \frac{1}{2^\kappa}, \frac{2}{2^\kappa}, \ldots, \frac{2^\kappa - 1}{2^\kappa}\}$
  is denoted $\work(A) = \gamma - \lg \frac{A}{T} \in [\gamma, \infty]$,
  where $\gamma \in \mathbb{R}^+$
  is the \emph{bias} parameter of the protocol.
\end{definition}

For genesis, we set $\work(G) = 1$ (an arbitrary constant) by convention.
To simplify the analysis, we will set $\gamma = 0$.

\begin{definition}[Chain Intrinsic Work]
  The \emph{intrinsic work} of a chain
  $C$ is the sum of the
  intrinsic work of all blocks in $C$.
  It is denoted as $\work(C) = \sum_{B \in C}{\work(H(B))}$.
\end{definition}

\noindent
\myparagraph[Blockchain notation]
For chain $C$, we write $[\alpha] \lhd C$
to denote the block $C[i]$ of $C$ such that
$\work(C[{:}{i}]) < \alpha \leq \work(C[{:}{i + 1}])$.
If $\work(C) < \alpha$, then let $[\alpha] \lhd C = \bot$.
If $\alpha$ is negative, then $[\alpha] \lhd C$ is defined as
the block $C[i]$ of $C$ such that
%TODO: Where to put the equality?
$\work(C[{i}{:}]) > -\alpha \geq \work(C[{i + 1}{:}])$.
We use the slicing notation $[{\alpha}{:}{\beta}] \lhd C$ to denote
$C[{i}{:}{j}]$ where $i$ is the index of $[\alpha] \lhd C$
and $j$ is the index of $[\beta] \lhd C$ in $C$ respectively.
The notation $[{\alpha}{:}] \lhd C$ means $C[{i}{:}]$,
and the notation $[{:}\beta] \lhd C$ means $C[{:}{j}]$,
where $i$ and $j$ are defined with respect to $\alpha$ and $\beta$
respectively as above.
Any continuous chunk of blocks $C[{i}{:}{j}]$ is called a \emph{subchain} of $C$.
Given a block $B$, we denote by $B.x$ the sequence of transactions
included in $B$.
Given a chain $C$, we denote by $C.x$
the sequence of transactions in all the blocks of $C$ in order, namely
${\big\lVert}_{B \in C} B.x$.

In Algorithm~\ref{alg.backbone} we show the code of an honest party.
First, the party is constructed using the \constructor function (Line~\ref{alg-backbone.constructor}).
In every round, each party is executed by the environment using function \execute
(this is an artifact of the lockstep round-based nature of
our time model).

\import{./}{algorithms/algorithm-pow.tex}

The honest party begins each round with a certain value stored in his chain $C$.
We say that the honest party \emph{has} chain $C$ at this round. The party calls $\net.\receive()$ to get all
the chains from the network (Line~\ref{alg-backbone.receive}) and chooses the
``best'' chain among them. We say that the honest party
\emph{adopts} this chain. By $\Chain[P][][r]$, we denote the chain that was adopted
by party $P$ at round $r$.
This comparison for the ``best'' chain is performed by
function \maxvalid in Line~\ref{alg-backbone.maxvalid}, and
is the single point that we deviate from the original Bitcoin protocol.
Next, the honest party attempts to mine a block using the \pow function (Line~\ref{alg.pow}),
which also remains the same as the original protocol: He repeatedly tries to find a block $B$
that satisfies the \pow equation $H(B) < T$, where the target
$T \in \{0, \frac{1}{2^\kappa}, \frac{2}{2^\kappa}, \ldots, \frac{2^\kappa - 1}{2^\kappa}\}$ is a small real number in
the interval $[0, 1)$.
If a block is found, this block is broadcast\footnote{We use the term \emph{broadcast}
to mean the unreliable, best-effort anonymous manner of communication between honest parties that
guarantees message delivery from one honest party to all other honest parties.
This is called \emph{diffuse} in the Backbone series of works.}
to the network using function $\net.\broadcast()$.

\import{./}{algorithms/algorithm-maxvalid.tex}

We will now analyze the functionality \maxvalid. The method receives as input
a set of chains and returns the ``best'' chain based on a validation and
chain adoption rule. The function iterates over all provided chains
and first checks their validity in Line~\ref{alg-maxvalid:validate}, using
function \validate (Algorithm~\ref{alg.validate}). The \validate function remains unchanged compared to
Bitcoin.
The chains that satisfy the validation rule
are compared with one another to find the chain
with the most intrinsic work (hereforth ``heaviest chain'').
Finally, in Line~\ref{alg-maxvalid:return}, we return the
``best'' chain $C_{\mmax}$.

\import{./}{algorithms/algorithm-validate.tex}

When the time comes to report the stable chain
(function $\lread$ in Algorithm~\ref{alg.backbone} Line~\ref{alg-backbone.read}),
after the function \execute has been called,
the honest party removes the unstable part of the chain, namely
the last $k$ bits of work from the chain, and reports
the remaining chain as stable. Note that, contrary to Bitcoin,
the variable $k$ is measured in \emph{bits of work}, and not in blocks
(looking ahead, $k$ will be shown to be polynomial in the
security parameter $\kappa$, and we will calculate its value
in the analysis section).

This concludes the \poem construction.

%\import{./}{algorithms/algorithm-operators.tex}
