\section{Construction}
In our construction, we only modify the chain adoption rule
of the original Bitcoin protocol.
In the original protocol, each honest party starts a round with a local chain
$C$ and checks for incoming messages. If a longer valid chain is received, then
the party adopts it. This comparison is done using function \maxvalid,
which returns the ``best" chain when given a set of chains as input.
In Algorithm~\ref{alg.backbone} Line~\ref{alg-backbone.maxvalid}, the
\maxvalid function is provided with chain $C$ and all the other chains received at that
round ($\bar M[i])$.

In Algorithm~\ref{alg.maxvalid}, we modify the \maxvalid function to return the valid
chain with the most intrinsic work instead of the longest one.
In \maxvalid, the same chain validation rules apply as before. In order to
validate a chain, in Line~\ref{alg-maxvalid:validate}, the \validate function is used.
The \validate function (Algorith~\ref{alg.validate}) remains unchanged compared to
the original Bitcoin protocol.


%\import{./}{algorithms/algorithm-operators.tex}
\import{./}{algorithms/algorithm-validate.tex}
\import{./}{algorithms/algorithm-maxvalid.tex}
\import{./}{algorithms/algorithm-honest.tex}
\import{./}{algorithms/algorithm-pow.tex}
