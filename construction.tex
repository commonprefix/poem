\section{Construction}
The protocol construction is similar to the original Bitcoin construction.
The \emph{chain validation} predicate remains unchanged, however we introduce
a few changes in the chain adoption rule.
Each miner starts a round with a local chain $C$ and checks for incoming messages
using function $\receive()$. If a better chain $\widetilde{C}$ is received, then the
miner adopts the new chain $\widetilde{C}$ at that round. The comparison is done using
function $\maxvalid()$, which return the ``best" chain
when given a set of chains as input. In the original Bitcoin protocol, the
``best'' chain is defined as the longest one; hence $\maxvalid()$ returns
the longest valid chain. In our construction, we define the ``best" chain being
the one with the most intrinsic difficulty.
Therefore, we propose $maxvalid()$ returning the valid chain
with the maximum intrinsic difficulty, as illustrated in Algorithm~\ref{alg.maxvalid}.
%TODO: Define (block and chain) intrinsic difficulty beforehand

\import{./algorithms/}{algorithm-maxvalid.tex}
