\section{Construction}
In the \poem construction, only the fork choice rule
of the original Bitcoin protocol is modified.
Honest parties, instead of adopting the longest chain, at the beginning
of each round, now adopt the chain with the most intrinsic work.

We let $\kappa$ be the security parameter and we
model our hash function $H$ as a random oracle.

A \emph{chain} $C$ is a sequence of blocks beginning with a designated
block $G$, the genesis block.

\begin{definition}[Block Intrinsic Work]
  The \emph{intrinsic work} of a block
  $B$ is denoted as $\work(B) = -\lg H(B)$.
\end{definition}

\begin{definition}[Chain Intrinsic Work]
  The \emph{intrinsic work} of a chain
  $C$ is the sum of the
  intrinsic work of all blocks in $C$.
  It is denoted as $\work(C) = \sum_{B \in C}{\work(B)}$.
\end{definition}

\noindent
\myparagraph[Blockchain notation]
For chain $C$, we write $[\alpha] \lhd C$
to denote the $i^\text{th}$ block of $C$ such that
$\work(C[{:}{i - 1}]) < \alpha \leq \work(C[{:}{i}])$.
If $\work(C) < \alpha$, then $[\alpha] \lhd C = \bot$.
If $\alpha$ is negative, then $[\alpha] \lhd C$ is defined as
the $i^\text{th}$ block of $C$ such that
$\work(C[{i - 1}{:}]) < -\alpha \leq \work(C[{i}{:}])$.
We write $[{\alpha}{:}{\beta}] \lhd C$ to denote
$C[{i}{:}{j}]$ where $i$ is the index of $[\alpha] \lhd C$
and $j$ is the index of $[\beta] \lhd C$ in $C$ respectively.
The notation $[{\alpha}{:}] \lhd C$ means $C[{i}{:}]$,
and the notation $[{:}\beta] \lhd C$ means $C[{:}{j}]$,
where $i$ and $j$ are defined with respect to $\alpha$ and $\beta$
respectively as above.
Given a block $B$, we denote by $B.x$ the sequence of transactions
included in $B$.
Given a chain $C$, we denote by $C.x$
the sequence of transactions in all the blocks of $C$ in order, namely
${\big\lVert}_{B \in C} B.x$.

In Algorithm~\ref{alg.backbone} we show the code of an honest party.
First, the party is constructed using the \constructor function (Line~\ref{alg-backbone.constructor}).
In every round, each party is executed by the environment using function \execute.

At the beginning of each round, the honest party calls $\net.\receive()$ to get all
the chains from the network (Line~\ref{alg-backbone.receive}). If a ``better''
valid chain is found, then the party adopts it.
This comparison for the ``better'' chain is performed by
function \maxvalid in Line~\ref{alg-backbone.maxvalid}, and
is the single point that we deviate from the original Bitcoin protocol.
Next, the honest party attempts to mine a block using the \pow function (Line~\ref{alg.pow}),
which also remains the same as the original protocol.

\import{./}{algorithms/algorithm-honest.tex}
\import{./}{algorithms/algorithm-pow.tex}

We will now analyze the functionality of \maxvalid. The method receives as input
a set of chains and returns the ``best'' chain based on a validation and
chain adoption rule. The function iterates over all the provided chains
and first checks their validity in Line~\ref{alg-maxvalid:validate}, using
function \validate (Algorithm~\ref{alg.validate}. The \validate function remains unchanged compared to
the original Bitcoin protocol.
The chains that satisfy the validation rule,
are compared with one another in order to find the chain
with the most intrinsic work. Finally, in Line~\ref{alg-maxvalid:return}, we return the
``best'' chain $C_{\mmax}$.

This concludes the \poem construction.

\import{./}{algorithms/algorithm-maxvalid.tex}
\import{./}{algorithms/algorithm-validate.tex}



%\import{./}{algorithms/algorithm-operators.tex}
