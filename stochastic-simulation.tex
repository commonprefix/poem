\section{Stochastic Simulation}
In these section we compare Bitcoin's and PoEM's latency experimentally.
We simulate various executions of Bitcoin and PoEM with different parameterizations.
In each simulation, we fix the block production rate $g$ and adversarial ratio $\beta$, and we
measure the latency of the system.
The latency of the system is the rate at which blocks get confirmed.
We use the private mining attack as the adversarial strategy, which was
proven~\cite{eiar} to be the best possible attack against Bitcoin in the continuous-time domain~\cite{bitcoin-made-simple}.
This means that the adversary mines blocks in private on her own chain, whereas the honest parties mine
their own blocktree, following the heaviest chain rule (in Bitcoin) or the most intrinsic work rule (in PoEM) respectively.
The adversary imposes a network delay of $\Delta$ to honest parties.
% TODO: Prove that the adversary's strategy is optimal.

Instead of performing simulations were proof of work takes place, we can simulate it by following a stochastic process.
We first simulate the execution of honest parties. We observe that the time between the creation of two blocks is
exponentially distributed with rate $g$. Hence, we sample from $\exp(g)$ to get the time that each honest block in the
execution was created. In Bitcoin, the work of each block is equal to one, whereas the work of a PoEM block is exponentially
distributed with rate $\frac{1}{\ln2}$. Hence, in PoEM's case, we must sample from $\exp(\frac{1}{\ln2})$ to get the work
of each block in the execution. The creation time of each block and its work are enough to simulate the honest parties' execution
and determine the blocktree that was constructed.

Then, independently we simulate the adversary's execution. Like in the honest execution, we sample from $\exp(g\frac{1 - beta}{beta})$
to get the time that each block in the adversary's execution was created, where $g\frac{1 - beta}{beta}$ is the adversary's block production rate.
Then, we sample from $\exp(\frac{1}{\ln2})$ to get the work of each block in the execution.
Since the adversary has no network delay, all her blocks are chained in series.

Having simulated the honest and adversary executions, we can now determine the latency of the system.
To do this, we determine the last point in time when the adversary had a chain with more work than the honest parties,
and we record the work $k$ of the honest chain that surpasses the adversary's chain immediately after that time.
For this execution, any confirmation parameter larger or equal to $k$ would keep the adversary from violating Common Prefix.
Hence, the latency of the system is $\frac{k}{d}$, were $d$ is the time it takes to produce $k$ work.